\documentclass{article}\usepackage[]{graphicx}\usepackage[]{color}
%% maxwidth is the original width if it is less than linewidth
%% otherwise use linewidth (to make sure the graphics do not exceed the margin)
\makeatletter
\def\maxwidth{ %
  \ifdim\Gin@nat@width>\linewidth
    \linewidth
  \else
    \Gin@nat@width
  \fi
}
\makeatother

\definecolor{fgcolor}{rgb}{0.345, 0.345, 0.345}
\newcommand{\hlnum}[1]{\textcolor[rgb]{0.686,0.059,0.569}{#1}}%
\newcommand{\hlstr}[1]{\textcolor[rgb]{0.192,0.494,0.8}{#1}}%
\newcommand{\hlcom}[1]{\textcolor[rgb]{0.678,0.584,0.686}{\textit{#1}}}%
\newcommand{\hlopt}[1]{\textcolor[rgb]{0,0,0}{#1}}%
\newcommand{\hlstd}[1]{\textcolor[rgb]{0.345,0.345,0.345}{#1}}%
\newcommand{\hlkwa}[1]{\textcolor[rgb]{0.161,0.373,0.58}{\textbf{#1}}}%
\newcommand{\hlkwb}[1]{\textcolor[rgb]{0.69,0.353,0.396}{#1}}%
\newcommand{\hlkwc}[1]{\textcolor[rgb]{0.333,0.667,0.333}{#1}}%
\newcommand{\hlkwd}[1]{\textcolor[rgb]{0.737,0.353,0.396}{\textbf{#1}}}%

\usepackage{framed}
\makeatletter
\newenvironment{kframe}{%
 \def\at@end@of@kframe{}%
 \ifinner\ifhmode%
  \def\at@end@of@kframe{\end{minipage}}%
  \begin{minipage}{\columnwidth}%
 \fi\fi%
 \def\FrameCommand##1{\hskip\@totalleftmargin \hskip-\fboxsep
 \colorbox{shadecolor}{##1}\hskip-\fboxsep
     % There is no \\@totalrightmargin, so:
     \hskip-\linewidth \hskip-\@totalleftmargin \hskip\columnwidth}%
 \MakeFramed {\advance\hsize-\width
   \@totalleftmargin\z@ \linewidth\hsize
   \@setminipage}}%
 {\par\unskip\endMakeFramed%
 \at@end@of@kframe}
\makeatother

\definecolor{shadecolor}{rgb}{.97, .97, .97}
\definecolor{messagecolor}{rgb}{0, 0, 0}
\definecolor{warningcolor}{rgb}{1, 0, 1}
\definecolor{errorcolor}{rgb}{1, 0, 0}
\newenvironment{knitrout}{}{} % an empty environment to be redefined in TeX

\usepackage{alltt}
\usepackage{amsmath}
\usepackage{amsthm}
\usepackage{amsthm}
\usepackage{amssymb}
\usepackage{longtable}
\usepackage{alltt}
\usepackage{setspace,relsize} % for latex(describe()) (Hmisc package) and latex.table
\usepackage{moreverb}
\usepackage[pdftex]{lscape}
\usepackage{lipsum}
\usepackage{tabularx}

 
\setlength{\oddsidemargin}{0.truein}
\setlength{\evensidemargin}{0.truein}
\setlength{\textwidth}{6.5truein} \setlength{\topmargin}{-.55truein}
\setlength{\textheight}{9.0truein}


\def\eqalign#1{\null\,\vcenter{\openup\jot\ialign
{\strut\hfil$\displaystyle{##}$&$\displaystyle{{}##}$\hfil\crcr#1\crcr}}\,}
\def\refhg{\hangindent=20pt\hangafter=1}
\def\refmark{\par\vskip 2mm\noindent\refhg}
\def\bLambda{\mbox{\boldmath $\Lambda$}}
\def\bDelta{\mbox{\boldmath $\Delta$}}
\def\btheta{\mbox{\boldmath $\theta$}}
\def\bSigma{\mbox{\boldmath $\Sigma$}}
\def\bbeta{\mbox{\boldmath $\beta$}}
\def\bphi{\mbox{\boldmath $\phi$}}
\def\balpha{\mbox{\boldmath $\alpha$}}
\def\bJ{{\bf J}}
\def\bgamma{\mbox{\boldmath $\gamma$}}
\def\bmu{\mbox{\boldmath $\mu$}}
\def\bY{{\bf Y}}
\def\bU{{\bf U}}
\def\bx{{\bf x}}
\def\bX{{\bf X}}
\def\bW{{\bf W}}
\def\bZ{{\bf Z}}
\def\bz{{\bf z}}
\def\logit{\rm logit}

\def\bgn{\begin{eqnarray*}}
\def\edn{\end{eqnarray*}}
\def\bg{\begin{eqnarray}}


% Fix latex
\def\smallskip{\vskip\smallskipamount}
\def\medskip{\vskip\medskipamount}
\def\bigskip{\vskip\bigskipamount}
 
% Hand made theorem
\newcounter{thm}[section]
\renewcommand{\thethm}{\thesection.\arabic{thm}}
\def\claim#1{\par\medskip\noindent\refstepcounter{thm}\hbox{\bf \arabic{chapter}.\arabic{section}.\arabic{thm}. #1.}
\it\ %\ignorespaces
}
\def\endclaim{
\par\medskip}
\newenvironment{thm}{\claim}{\endclaim}

\usepackage[utf8]{inputenc}
%\usepackage[enlgish]{babel}
 
\newtheorem{theorem}{Theorem}
\IfFileExists{upquote.sty}{\usepackage{upquote}}{}
\begin{document}
%\SweaveOpts{concordance=TRUE}

\begin{center}
{\Large {\bf Nana Boateng}} \\
\vspace{5mm}
{\Large Department of Mathematics} \\
\vspace{5mm}
%{\Large MATH 8764} \\
\vspace{5mm}
{\Large EM with Marginal Compactibility} \\
\end{center}

\section{1}
Let $X_{1},X_{2},\cdots,X_{n}$ be binary exchangeable. \\
Let $\lambda_{k}=P(X_{1}=X_{2}=\cdots=X_{k}=1|N=n)$\\

$P(X_{1}=x_{1},\cdots,X_{n}=x_{n})=P(X_{1}=1,\cdots,X_{r}=1,X_{r+1}=0,\cdots,X_{n}=0)$\\

$=E[X_{1}X_{2},\cdots,X_{r}(1-X_{r+1},\cdots,(1-X_{n}))]$\\

$=\sum_{k=0}^{n-r}(-1)^{k}\binom{k}{n-r}.1.P(X_{1}=X_{2}=,\cdots,=X_{r+k}=1)$\\

Let $r=\sum_{i=1}^{n-r}x_{i}$ then\\

$P(R=r)=\binom{n}{r}\sum_{k=0}^{n-r}(-1)^{k}\binom{k}{n-r}\lambda_{r+k,n}$\\

\newtheorem{mydef}{Definition}
\begin{mydef}
The $M\ddot{o}bius$ function isdefined as:\\
$\mu(n) = \left\{ \begin{array}{rcl}
(-1)^{k}   \mbox{\hspace{5mm}if  $n=p_{1},\cdots,p_{k}$ ,for $1\leq i \leq j \leq k$,$p_{i} \neq p_{j}$}
& \\ 0& \mbox{otherwise}  &\\

\end{array}\right.$\\
\end{mydef}

\begin{theorem}

Let $n\geq 1$ be an integer\\

$\sum_{d \setminus n}\mu(d) = \left\{ \begin{array}{rcl}
1   \mbox{\hspace{5mm}if  $n=0$}
& \\ 0& \mbox{otherwise } &\\

\end{array}\right.$\\
\end{theorem}

\begin{theorem}[$M\ddot{o}bius Inversion Theorem$]


If $f(n)=\sum_{d \setminus n}g(d)$\\

    $g(n)=\sum_{d \setminus n}\mu(\frac{n}{d})f(d)$\\
      
\end{theorem}

By the above theorem we can deduce that\\

$\lambda_{r,n}=\frac{\sum_{j=0}^{n-r}\mu(j)}{\binom{n}{n-j}}P_{n-j,n} $\\





The conditional distribution of the missing data $S$ is\\


$P_{r}(S=s|R=r,\Theta^{(t)})=\frac{P([s\hspace{2mm} out \hspace{0.5mm} of \hspace{2mm} M-n] \&[r\hspace{2mm} out \hspace{0.5mm}of \hspace{2mm}n])}{P(r\hspace{2mm} out\hspace{0.5mm} of \hspace{2mm}n|\Theta^{(t)})}$


$=\frac{P(r+s\hspace{2mm} out \hspace{0.5mm} of \hspace{2mm} M)P(r\hspace{2mm} out \hspace{0.5mm}of \hspace{2mm}n|r+s \hspace{2mm} out \hspace{.5mm} of \hspace{2mm}M )}{P(r\hspace{2mm} out\hspace{0.5mm} of \hspace{2mm}n|\Theta^{(t)})}$\\

$\frac{P^{(t)}_{r+s}\frac{\binom{M-r-s}{n-r}\binom{r+s}{r}}{\binom{M}{n}}}{P^{(t)}_{r,n}}$\\


$\frac{P^{(t)}_{r+s}\binom{n}{r}\binom{M-n}{s}}{\binom{M}{r+s}P^{(t)}_{r,n}}$\\


$Q(\theta | \theta^{t})=E_{s|R=r,\theta^{(t)}}\log P_{r+s,M}$\\

$\sum_{s=0}^{M-n}\frac{\binom{n}{r}\binom{M-n}{s}  P_{r+s,M}^{(t)}}{\binom{M}{r+s}  P_{r,n}}\log P_{r+s,M}$\\

Let $k=r+s$,Then \\

$Q(\theta | \theta^{t})=\frac{\sum_{k=r}^{M-n+r}\binom{n}{r}\binom{M-n}{k-r}P_{k,M}^{(t)}\log P_{k,M}}{\binom{M}{k}P_{r,n}^{(t)}}$\\



$=\frac{\sum_{k=0}^{M}\binom{n}{r}\binom{M-n}{k-r}P_{k,M}^{(t)}\log P_{k,M}}{\binom{M}{k}P_{r,n}^{(t)}}$\\



\end{document}




