%\documentclass[a4paper,10pt]{article}
\documentclass[12pt]{article}
\usepackage{graphicx}
\usepackage{amssymb}
\usepackage{amsmath}
\usepackage{longtable}
\usepackage{wrapfig}
%\usepackage{color}
%\usepackage{amsthm}
\usepackage[utf8]{inputenc}
\usepackage[T1]{fontenc}
\usepackage{lmodern}
\usepackage{listings}
\usepackage[usenames,dvipsnames]{color}
\usepackage{fancyhdr}
\usepackage{fullpage}
%\usepackage[top=tlength, bottom=blength, left=llength, right=rlength]{geometry}
%\usepackage{geometry}
%\usepackage[a4paper]{geometry}
\definecolor{MyDarkGreen}{rgb}{0.0,0.4,0.0}
%\usepackage[pdftex]{graphicx}
%\usepackage{mathtools}
\usepackage{listings}
\lstset{language=C++}
%===========================================================================================
% Numbering Equations
%\numberwithin{equation}{section} %sets equation numbers <chapter>.<section>.<index>
\numberwithin{equation}{subsection} %sets equation numbers <chapter>.<section>.<subsection>.<index>
%\numberwithin{equation}{subsubsection} %sets equation numbers <chapter>.<section>.<subsection>.<subsubsection>.<index>


%===================================================================================================================
% Basic Commands for Page settings,Chapters, Appendices, Sections, etc..
%====================================================================================================================
%\setlength{\oddsidemargin}{.5in} \setlength{\topmargin}{0in}
%\setlength{\headheight}{.2in} \setlength{\headsep}{.2in}
%\setlength{\textwidth = 6.0in} \setlength{\textheight = 8.3in}

\setlength{\oddsidemargin}{0.6in} \setlength{\topmargin}{-0.3in} %adjust side margins and top margins
\setlength{\headheight}{.2in} \setlength{\headsep}{.2in}
\setlength{\textwidth = 6.0in} \setlength{\textheight = 8.3in}



\def\thebiblio#1{\list
{[\arabic{enumi}]}{\settowidth\labelwidth{[#1]}\leftmargin\labelwidth
\advance\leftmargin\labelsep
\usecounter{enumi}}
\def\newblock{\hskip .11em plus .33em minus .07em}
\sloppy\clubpenalty4000\widowpenalty4000
\sfcode`\.=1000\relax}
\let\endthebiblio=\endlist

\newcommand{\sect}[1]{% Basic settings for general chapters
\cleardoublepage
\clearpage
\newpage
\begin{center}
\addtocounter{section} {1}
\setcounter{subsection} {0}
\section* {\normalsize \bf{CHAPTER \thesection \\ #1}}
\addcontentsline{toc}{section}{CHAPTER
\protect\numberline{\thesection : } #1\dotfill}
\end{center}
\thispagestyle{myheadings} }

\newcommand{\appen}[1]{% Basic settings for appendix chapters
\cleardoublepage
\clearpage
\newpage
\begin{center}
\addtocounter{section} {1}
\renewcommand{\thesection}{\Alph{section}}
\setcounter{subsection} {0}
\setcounter{table}{0}
\section* {\normalsize \bf{APPENDIX \thesection \\ #1}}
\addcontentsline{toc}{section}{APPENDIX
\protect\numberline{\thesection : } #1\dotfill}
\end{center}
\renewcommand{\thesubsection}{\Alph{section}.\arabic{subsection}}
\renewcommand{\thesubsubsection}{\Alph{section}.\arabic{subsection}.\arabic{subsubsection}}
\renewcommand{\theequation}{\Alph{section}.\arabic{equation}}
\renewcommand{\thetable}{\Alph{section}.\arabic{table}}

\thispagestyle{myheadings} }

%===================================================================================================================
% Basic Commands for Theorem, Lemma,Proposition,etc.....

\renewcommand{\baselinestretch}{2}
\renewcommand{\arraystretch}{.5}
\newcommand{\qed}{\hfill$\Box$}
\newtheorem{fact}{Theorem}[section]
\newtheorem{claim}{Claim}
\newtheorem{theorem}[fact]{Theorem}
\newtheorem{word}[fact]{Definition}
\newtheorem{prop}[fact]{Proposition}
\newtheorem{ob}[fact]{Observation}
\newtheorem{Corollary}[fact]{Corollary}
\newtheorem{corollary}[fact]{Corollary}
\newtheorem{lemma}[fact]{Lemma}
\newtheorem{Guess}[fact]{Conjecture}
\newtheorem{conj}[fact]{Conjecture}
\def\theotheorem{A\arabic{theorem}}
\newtheorem{mydef}{Definition}
%\newtheorem{theorem}{Theorem}[section]
%\newtheorem{lemma}[theorem]{Lemma}
%\newtheorem{proposition}[theorem]{Proposition}
\newtheorem{thm}{Theorem}
\newtheorem{lem}{Lemma}[thm]
%\newtheorem{corollary}[theorem]{Corollary}
%\newtheorem{cor}[theorem]{Corollary}
\newenvironment{proof}[1][Proof]{\begin{trivlist}
\item[\hskip \labelsep {\bfseries #1}]}{\end{trivlist}}
\newenvironment{definition}[1][Definition]{\begin{trivlist}
\item[\hskip \labelsep {\bfseries #1}]}{\end{trivlist}}
\newenvironment{example}[1][Example]{\begin{trivlist}
\item[\hskip \labelsep {\bfseries #1}]}{\end{trivlist}}
\newenvironment{remark}[1][Remark]{\begin{trivlist}
\item[\hskip \labelsep {\bfseries #1}]}{\end{trivlist}}
%================================================================================
%%%%%%%%%%%commands for problem
%================================================================================
\makeatletter
\newenvironment{problem}{\@startsection
       {section}
       {1}
       {-.2em}
       {-3.5ex plus -1ex minus -.2ex}
       {2.3ex plus .2ex}
       {\pagebreak[3]%forces pagebreak when space is small; use \eject for better results
       \large\bf\noindent{Problem }
       }
       }
       {%\vspace{1ex}\begin{center} \rule{0.3\linewidth}{.3pt}\end{center}}
       \begin{center}\large\bf \ldots\ldots\ldots\end{center}}
\makeatother


%
%Fancy-header package to modify header/page numbering
\pagestyle{fancy}
%\addtolength{\headwidth}{\marginparsep} %these change header-rule width
%\addtolength{\headwidth}{\marginparwidth}
\lhead{Problem \thesection} \chead{} \rhead{\thepage}
\lfoot{\small\scshape course name} \cfoot{} \rfoot{\footnotesize
PS\#}
\renewcommand{\headrulewidth}{0.3pt}
\renewcommand{\footrulewidth}{.3pt}
%\setlength\voffset{-0.25in} \setlength\textheight{648pt} \maketitle
%\maketitle
 %\thispagestyle{empty}



%===========================================================================
% commands for displaying codes in appendix
%============================================================================
\lstloadlanguages{Matlab}%
\lstset{language=Matlab,                        % Use MATLAB
        %frame=single,                           % Single frame around code
        basicstyle=\small\ttfamily,             % Use small true type font
        keywordstyle=[1]\color{Blue}\bf,        % MATLAB functions bold and blue
        keywordstyle=[2]\color{Purple},         % MATLAB function arguments purple
        keywordstyle=[3]\color{Blue}\underbar,  % User functions underlined and blue
        identifierstyle=,                       % Nothing special about identifiers
                                                % Comments small dark green courier
        commentstyle=\usefont{T1}{pcr}{m}{sl}\color{MyDarkGreen}\small,
        stringstyle=\color{Purple},             % Strings are purple
        showstringspaces=false,                 % Don't put marks in string spaces
        tabsize=5,                              % 5 spaces per tab
        %
        %%% Put standard MATLAB functions not included in the default
        %%% language here
        morekeywords={xlim,ylim,var,alpha,factorial,poissrnd,normpdf,normcdf},
        %
        %%% Put MATLAB function parameters here
        morekeywords=[2]{on, off, interp},
        %
        %%% Put user defined functions here
        morekeywords=[3]{FindESS, homework_example},
        %
        morecomment=[l][\color{Blue}]{...},     % Line continuation (...) like blue comment
        numbers=left,                           % Line numbers on left
        firstnumber=1,                          % Line numbers start with line 1
        numberstyle=\tiny\color{Blue},          % Line numbers are blue
        stepnumber=5                            % Line numbers go in steps of 5
        }

% Includes a MATLAB script.
% The first parameter is the label, which also is the name of the script
%   without the .m.
% The second parameter is the optional caption.
\newcommand{\matlabscript}[2]
  {\begin{itemize}\item[]\lstinputlisting[caption=#2,label=#1]{#1.m}\end{itemize}}





\begin{document}
\pagestyle{empty}
\pagenumbering{roman}

%================================================ Approval Page ===================================================


\newpage
%================================================ Title Page ======================================================
\begin{center}
\thispagestyle{empty} % takes out page numbering
  {ADVANCED MACROECONOMICS\\ [.07in]} \rm
\rule{1.25in}{.01in}\\[.0 in]
\today
\vspace{.6in}



\vspace{.6in}

\vspace{.6in}

by  \\ [.06in]
{Nana  Akwasi Abayie Boateng} \\[.06in]
\end{center}
\author{Nana Akwasi Abayie Boateng\footnotemark[1]}












\def\R{\mathbb{R}}
\def\N{\mathbb{N}}
\def\Z{\mathbb{Z}}
\def\Q{\mathbb{Q}}
\def\la{\langle}
\def\ra{\rangle}

\def\dist{{\rm dist}}
\def\X{{\bf X}}
\def\C{{\bf C}}
\def\D{{\bf D}}
\def\I{{\bf I}}
\def\J{{\bf J}}
\def\x{{\bf x}}
\def\y{{\bf y}}
\def\z{{\bf z}}
\def\W{{\bf W}}
\def\g{{\bf g}}
\def\e{{\bf e}}
\def\b{{\bf b}}
\def\u{{\bf u}}
\def\Beta{{\bf \beta}}
\def\pen{{\rm pen}}
\def\argmin{{\rm argmin}}
\def\diag{{\rm diag}}
\def\sgn{{\rm sgn}}
\def\supp{{\rm\rm supp}}


\vspace*{1cm}
%============================================= Appendix Separation Page  ===============================================
\newcommand{\Appendixpage}{
    \setcounter{section}{0}
    \renewcommand{\baselinestretch}{1}\small\normalsize
    \thispagestyle{myheadings}
    \addcontentsline{toc}{section}{APPENDICES\dotfill}
    \mbox{}
    \vfil
    \begin{center}%
    APPENDICES
    \vfil
    \end{center}%
    \renewcommand{\baselinestretch}{1.66} \small\normalsize%
    \cleardoublepage
  }

%================================================ Chapter 1 ==============================================================

\newpage
\section{Question 1}
\subsection{Decentralized Version of the Neoclassical Growth Model}
%section{Descentralized version of the Neoclassical growth model}
\subsection{Firm}
\begin{equation} \label{eqGEW}
Y_t=F(K_{t,}A_{t }N_{t})
\end{equation}
\begin{enumerate}
  \item {Firm's maximization Problem}
$$
 \underset{(k_{t},N_{t})}{max} \Pi=F(K_{t},A_{t }N_{t}) - r_{t} K_{t}- \omega_{t} \tilde{N_{t}}
$$

FOC
$$
\frac{\partial \Pi}{\partial K}=F_{k}(K_{t},A_{t} N_{t}) - r_{t }=0
$$

$$
F_{K}= r_{t}
$$

$$
\frac{\partial \Pi}{\partial N_{t}}=F_{N}(K_t,A_{t} N{_t}) - w_t =0
$$

$$
F_{N}=\frac{ W_t}{A_t}
$$
\item {illustration of  equilibruim profits to be  zero}\\


Based on Euler's theorem
$$
F(K,L)= KF_{k} + LF_{L}
$$

$$
F(K_t,A_t N_t)= KF_{k }+ANF_{N}
$$

$$
= Kr_{t} + AN\frac{W_{t}}{A}
$$

$$
= Kr_{t }+N\omega_{t}
$$

$$
\Pi= Kr_{t} + N\omega_{t} - rK_{t }- \omega_{t }N_{t}
$$

$$
\Pi= 0
$$

\subsection{Household}
\item \textbf{Let's set up the  Lagrangean} \\
The household maximization problem is

$$
\underset{K_{t}}{max} \sum_{t=0}^{\infty} \beta^t U(c_t)
$$


s.t
$$
C_{t }+I_{t}= r_{t }K_{t }+ \omega_{t }N_{t}  +\Pi_{t}
$$

$$
K_{t+1}= (1-\delta)K_{t }+I_{t}
$$
by substitution we will have one constraint
$$
K_{t+1}= ( r+1-\delta) K_{t }+ \omega_{t} N_{t }+\pi_{t} - C_{t}
$$
\item The FOC \\

$$
L=\sum_{t=0}^{\infty} \beta^{t }U(c_{t}) + \sum_{t=0}^{\infty} \lambda_{t}[  ( r+1-\delta) K_{t} + \omega_{t} N_{t} +\pi_{t} - C_{t} -K_{t+1}]
$$

$$
[C_t] : \beta^t U'(C_{t})- \lambda_t =0
$$

$$
[K_{t+1}]: -\lambda_{t }+ \lambda_{t+1} ( r+1-\delta) =0
$$

$$
[\lambda_{t}]:  ( r+1-\delta) K_{t} + \omega_{t }N_{t} +\pi_{t} - C_{t} -K_{t+1} =0
$$
\item \textbf{Euler Equation}\\
$$
[ C_{t+1}] : \beta^{t+1} U\prime( C_{t+1})= \lambda_{t+1}
$$
let replace  $\lambda_t$  and  $\lambda_{t+1}$ in equation 2 . We will have
$$
 \frac{U\prime( C_t)}{\beta U\prime( C_{t+1} }= r_{t }+1-\delta
$$
\subsection{Dynamic competitive equilibrium}
%\subsubsection{Characterization of equilibrium}
\item {Three equations to characterize equilibrium}\\

$$
  F( K_{t}, A_{t} N_{t})
$$

$$
F \left( \frac{ K_{t}}{A_{t} N_{t}, 1} \right)= F(k,1)= F(k)
$$
$$
\Pi= F(k)- r_t k_t - \frac{\omega_{t}}{A_{t}}
$$
FOC
$$
F\prime(k)= r_{t}
$$
$$
F(N)\prime=0
$$
$$
 \frac{U\prime( C_{t})}{\beta U\prime( C_{t+1} }= r_{t} +1-\delta
$$
$$
Lom:  ( r+1-\delta) K_{t }+ \omega_{t }N_{t }+\pi_{t }- C_{t }-K_{t+1} =0.
$$
At the end we will have

$$
 \frac{U\prime( C_t)}{\beta U^( C_{t+1} }= f(k)\prime +1-\delta
$$

No the only difference here is that the growth rate $\gamma(A)$ and $\gamma(N)$ is zero 

\item  \textbf{Equilibrium Wage} 
$$
\Pi =0  \implies \Pi= f(k)- r_{t} k_{t}- \frac{w_{t}}{A_{t}}
$$

$$
\omega_{t}=A_{t}( r_{t} k_{t}-f(k)
$$
\end{enumerate}
1.1\\
1\\
$\underset{(Y_{T},K_{T},N_{T})}{\max} |^{\infty}_{t=0} \sum_{t=0}^{\infty} p_{t}(Y_{t}-\omega_{t}N_{t}-r_{t}K_{t})$\\

s.t.\\

$Y_{t}=F(K_{t},A_{t}N_{t})$\\

By setting $p_{t}=1$ in the last period,we have\\

$\underset{K,N}{\max}=F(K,AN)-rK-\omega N$\\

This gives the first order conditions

$F_{K}(K,AN)-r=0$\\
$F_{N}=r$\\
$F_{N}(K,AN)-\omega=0$\\
$F_{N}=\omega$\\
The optimal choice of of capital  $K^{*}$  and $N^{*}$ is equal to  their  marginal product  which is equal to their prices.\

2\\


1.2\\
\subsection{Household}
$r_{t}K_{t}+\omega_{t}\tilde{N_{t}}+\pi_{t}$

$C_{t}+I_{t}=r_{t}K_{t}+\omega \tilde{N_{t}} +\pi_{t}$\\

The law of motion of capital is\\
$K_{t+1}=(1-\delta)\tilde{K_{t}}+I_{t}$\\

For a discount rate $0<\beta<1$ and initial capital endowment $K_{0}$,The household's utility maximization problem is \\

$\underset{C_{t},I_{t},K_{t}}{max}|_{t=0}^{\infty} \sum_{t=0}^{\infty}\beta^{t}u(c_{t})$\\
s.t.
$K_{t+1}=(1-\delta)\tilde{K_{t}}+I_{t}$\\
$r_{t}\tilde{K_{t}}+\omega_{t}\tilde{N_{t}}+\pi_{t}=C_{t}+I_{t}$\\
$K_{0}$ given\\


which simplifies to\\
$r_{t}\tilde{K_{t}}+\omega_{t}\tilde{N_{t}}+\pi_{t}-C_{t}-K_{}+(1-\delta)\tilde{K_{t}}=0$\\

Let the lagrange multiplier be $\lambda$ .Then the Lagrange multiplier is \\

$\mathcal{L}_{C_{t},I_{t},K_{t+1},\lambda}:=\sum_{t=0}^{\infty}\beta^{t}u(c_{t})+sum_{t=0}^{\infty}\lambda_{t}(r_{t}\tilde{K_{t}}+\omega_{t}\tilde{N_{t}}+\pi_{t}-C_{t}-K_{}+(1-\delta)\tilde{K_{t}})$

$\frac{\partial \mathcal{L}}{\partial C_{t}}=\beta ^{t}u^{\prime}(C_{t})-\lambda_{t}=0$

$\frac{\partial \mathcal{L}}{\partial K_{t+1}}=\lambda_{t+1}(r_{t+1}+(1-\delta)))-\lambda_{t}=0$

$\beta^{t+1} u^{\prime}(C_{t+1})=\lambda_{t+1}$

The Euler Equation   is obtained as\\
$\lambda_{t}=\lambda_{t+1}(r_{t+1}+(1-\delta)))$\\

$\beta^{t} u^{\prime}(C_{t})=\beta^{t+1} u^{\prime}(C_{t+1})(r_{t+1}+(1-\delta)))$\\

$\frac{u^{\prime}(C_{t})}{u^{\prime}(C_{t+1})}=\beta(r_{t+1}+(1-\delta))$

%\subsection{Dynamic Competitive Equilibrium}









\newpage
\section{Question 2}
\pagestyle{myheadings} \markboth{  } {  }
\pagenumbering{arabic}
%-------------------------------------------------------------------------------------------------------------------------
$U(c_{t},L_{t})=\mu \ln c_{t}+(1-\mu)\ln(\bar{L}-L_{t})$\\

1 $ \underset{(y_{t},K_{t},L_{t})}{\max}  \sum_{t=0}^{\infty}\beta^{t}U(c_{t},k_{t})$\\
s. t.\\
$y_{t}=c_{t}+i_{t}$\\

$i_{t}=y_{t}+-c_{t}$\\

LOM :$k_{t+1}+(1-s)k_{t}+i_{t}$\\

$y_{t}=c_{t}+k_{t+1}-(1-\delta)k_{t}$\\

$\mathcal{L}=\sum_{t=0}^{\infty}\beta^{t}\left(\mu \ln c_{t} +(1-\mu)\ln(\bar{L}-L_{t}) \right)+\sum_{t=0}^{\infty}\lambda_{t}\left((1-\delta)k_{t}-k_{t+1}-c_{t}+AK_{t}^{\alpha}L_{t}^{1-\alpha}\right)$\\

 %\begin{equation) 
 2\\

$[c_{t}]: \frac{\beta^{t}\mu}{c_{t}}-\lambda_{t}=0$\\

%\end{equation)

$[k_{t+1}]:=-\lambda_{t}+\lambda_{t+1}\left( (1-\delta)+\alpha Ak_{t+1}^{\alpha-1}L_{t+1}^{1-\alpha}   \right)$\\

$[L_{t}]: \frac{-\beta^{t}(1-\mu)}{\bar{L}-L_{t}}+\lambda_{t}(1-\alpha)Ak_{t}^{\alpha}L_{t}^{-\alpha}$\\

3\\
Eulers Equation\\
$\frac{\beta^{t}\mu}{c_{t}}=\frac{\beta^{t+1}\mu}{c_{t+1}}\left( 1-\delta+\alpha A k_{t+1}^{\alpha-1}L_{t+1}^{1-\alpha} \right)$\\

$\frac{c_{t+1}}{c_{t}}=\beta \left(1-\delta+ \alpha A k_{t+1}^{\alpha-1} L_{1-\alpha}^{t+1} \right)$\\


$k^{*}=\left (\frac{\frac{1}{\beta}-1+\delta}{A \alpha L^{*(1-\alpha)}}  \right)^{\frac{1}{\alpha-1}}$
The steady state capital is given by\\

$k^{*}=\left(\frac{A \alpha L^{*1-\alpha}}{\frac{1}{\beta} -1+\delta} \right)^{\frac{1}{1-\alpha}}$\\



%$\lambda_{t}=\lambda_{t+1}(1-\delta+\alpha k_{t+1}^{\alpha-1}l_{t+1}^{1-\alpha})$\\


%$\frac{c_{t+1}}{c_{t}}=\beta (1-\delta+\alpha k_{t+1}^{\alpha-1}l_{t+1}^{1-\alpha})$\\

$\beta^{t}(1-\mu)=\lambda_{t}(1-\alpha)Ak_{t}^{\alpha}L_{t}^{-\alpha}(\bar{L}-L_{t})$
%=\lamba_{t}(1-\alpha)Ak_{t}^{\alpha}L_{t}^{-\alpha}(\bar{L}-L_{t})$\\

%$\frac{\beta^{t}(1-\mu)}{\lamba_{t}(1-\alpha)Ak_{t}^{\alpha}}=\frac{\bar{L}-L_{t}}{L_{t}}$\\

$\frac{\beta^{t}(1-\mu)}{\lambda_{t}(1-\alpha)Ak_{t}^{\alpha}}=\frac{\bar{L}-L_{t}}{L_{t}^{\alpha}}$\\

$L_{t}^{\alpha}\frac{\beta^{t}(1-\mu)}{\lambda_{t}(1-\alpha)Ak_{t}^{\alpha}}+L_{t}-\bar{L}=0$\\

$L_{t}^{\alpha}\frac{\beta^{t}(1-\mu)}{\frac{\beta^{t}\mu}{c_{t}}(1-\alpha)Ak_{t}^{\alpha}}+L_{t}-\bar{L}=0$\\

The steady state labor\\

$L^{*\alpha}\left(\frac{C^{*}(1-\mu)}{\mu(1-\alpha)A k^{*\alpha}} \right)+L^{*}-\bar{L}=0$


$L^{*\alpha}\left( (\frac{1}{\mu}-1)\frac{c^{*}}{(1-\alpha)Ak^{*\alpha}}  \right)+L^{*}-\bar{L}=0$\\
The steady state consumption is \\
$c^{*}=(1-\delta)k^{*}-k^{*}+Ak^{*\alpha}L^{* 1-\alpha}$\\
$c^{*}=k^{*}-\delta k^{*}-k^{*}+Ak^{*\alpha}L^{* 1-\alpha}$\\
$c^{*}=-\delta k^{*}+Ak^{*\alpha}L^{* 1-\alpha}$\\

4\\
An increase in A raises the steady state consumption ,capital and consumption

5\\
An incease in the  the number of hours available $\bar{L}$ causes a decrease in the levels of steady state labor.The decrease in  steady state labor causes the steady state capital to also decrease.The  the decrease  in capital results in a decrease in consumption.









\newpage
\section{Question 3}

\begin{figure}[ht!]
\centering
\includegraphics[width=120mm,height=100mm]{31ia.png}
\caption{3.1.ii  Phase Diagram\label{overflow}}

3.1.i\\
$k_{t+1}=\frac{(1-\delta)k_{t}+sf(k_{t})}{(1+\gamma_{N})(1+\gamma_{A})}$\\

$(1+\gamma_{N})(1+\gamma_{A})k_{t+1}-(1-\delta)k_{t}=sf(k_{t})$\\

$(1+\gamma_{N})(1+\gamma_{A})k_{t+1}-(1-\delta)k_{t}=sk^{\alpha}_{t}$\\

$(1+\gamma_{N})(1+\gamma_{A})k^{*}-(1-\delta)k^{*}=sk^{* \alpha}$\\

$(1+\gamma_{N})(1+\gamma_{A})-(1-\delta)=sk^{* \alpha-1}$\\

$k^{\alpha-1}_{t}=\frac{(1+\gamma_{N})(1+\gamma_{A})-(1-\delta)}{s}$\\

$k^{*}=\left(\frac{(1+\gamma_{N})(1+\gamma_{A})-(1-\delta)}{s}\right)^{\frac{1}{\alpha-1}}$\\


$k^{*}=\left(\frac{s}{(1+\gamma_{N})(1+\gamma_{A})-(1-\delta)}    \right)^{\frac{1}{1-\alpha}}$\\



$k^{*gold}=3.4867$\\

$c^{*gold}=1.0182$\\

$i^{*gold}=0.43364$\\

$s^{*gold}=0.3$\\

\end{figure}
3.1.ii\\
$\frac{Y_{t}}{A_{t}N_{t}}=\frac{K_{t}^{\alpha}}{A_{t}N_{t}}(\frac{A_{t}N_{t}}{A_{t}N_{t}})^{1-\alpha}$\\
$y_{t}=k_{t}^{\alpha}$\\



\newpage
3.2.i\\
\begin{figure}[ht!]
\centering
\includegraphics[width=120mm,height=100mm]{32i.png}
\caption{3.1.i  growth of normalized capital stock\label{overflow}}
\end{figure}

3.2.i\\
$k_{t+1}=\frac {(1-\delta)k_{t}+sf(k_{t})}{(1+\gamma_{N})(1+\gamma_{A})}$\\






\newpage
3.2.ii\\
\begin{figure}[ht!]
\centering
\includegraphics[width=120mm,height=100mm]{32ii.png}
\caption{3.1.i  growth of non-normalized capital stock\label{overflow}}
\end{figure}
3.2.ii\\
$K_{t}=k_{t}((1+\gamma_{A})^{t}(1+\gamma_{N})^{t}$\\

$K_{t-1}=k_{t-1}((1+\gamma_{A})^{t-1}(1+\gamma_{N})^{t-1}$\\

\newpage
3.2.iii\\
\begin{figure}[ht!]
\centering
\includegraphics[width=120mm,height=100mm]{32iii.png}
\caption{3.1.i  growth of non-normalized capital stock\label{overflow}}
\end{figure}
3.2.iii\\
$Y_{t}=K_{t}^{\alpha}(A_{t}N_{t})^{1-\alpha}$\\

Non-normalized output\\
$Y_{t}=K_{t}^{\alpha}((1+\gamma_{A})^{t}(1+\gamma_{N})^{t}A_{0}N_{0})^{1-\alpha}$\\
%At the steady state
Non-normalized consumption\\
$C_{t}=(1-s)Y_{t}=(1-s)K_{t}^{\alpha}((1+\gamma_{A})^{t}(1+\gamma_{N})^{t}A_{0}N_{0})^{1-\alpha}$


Non-normalized Investment\\
$I_{t}=sY_{t}=sK_{t}^{\alpha}((1+\gamma_{A})^{t}(1+\gamma_{N})^{t}A_{0}N_{0})^{1-\alpha}$\\


\newpage
3.3.i\\
\begin{figure}[ht!]
\centering
\includegraphics[width=120mm,height=100mm]{33i.png}
\caption{3.1.i  normalized capital stock\label{overflow}}
\end{figure}
3.3.i\\
normalized capital stock\\
$k_{t+1}=\frac{(1-\delta)k_{t}+sf(k_{t})}{(1+\gamma_{A})(1+\gamma_{N})}$\\

$k_{t+1}=\frac{(1-\delta)k_{t}+sk_{t}^{\alpha}}{(1+\gamma_{A})(1+\gamma_{N})}$

normalized consumption\\
$C_{t}=(1-s)Y_{t}$\\
$C_{t}=(1-s)K_{t}^{\alpha}(A_{t}N_{t})^{1-\alpha}$\\

$\frac{C_{t}}{A_{t}N_{t}}=(1-s)\frac{K_{t}^{\alpha}}{A_{t}N_{t}}(\frac{A_{t}N_{t}}{A_{t}N_{t}})^{1-\alpha}$\\
$c_{t}=(1-s)k_{t}^{\alpha}$\\
\newpage
3.3.ii\\
\begin{figure}[ht!]
\centering
\includegraphics[width=120mm,height=100mm]{33ii.png}
\caption{3.1.i  normalized output \label{overflow}}
\end{figure}


3.3.ii
normalized output\\
$\frac{Y_{t}}{A_{t}N_{t}}=\frac{K_{t}^{\alpha}}{A_{t}N_{t}}(\frac{A_{t}N_{t}}{A_{t}N_{t}})^{1-\alpha}$\\
$y_{t}=k_{t}^{\alpha}$\\

\newpage
3.3.iii\\
\begin{figure}[ht!]
\centering
\includegraphics[width=120mm,height=100mm]{33iii.png}
\caption{3.1.i  normalized capital stock\label{overflow}}
\end{figure}


\newpage
3.3.iv\\
\begin{figure}[ht!]
\centering
\includegraphics[width=120mm,height=100mm]{33iv.png}
\caption{3.1.i  normalized capital stock\label{overflow}}
\end{figure}

\cleardoublepage

%=================================================BIBLIOGRAPHY==============================================================
%\newpage
%\begin{center}
%{\bf BIBLIOGRAPHY}
%\end{center}
%\addcontentsline{toc}{section}{\rm BIBLIOGRAPHY \dotfill}
%\begin{thebiblio}{99}
%---------------------------------------------------------------------------------------------------------------------------




%\bibitem{PSJ}
%Paul Wilmott,Sam Howson,Jeff Dewynne  \emph{The Mathematics of
%Financial Derivatives}. A Student Introduction, Cambridge university
%press.

%\bibitem{CF}
%Omur Ogur \emph{An Introduction to Computational Finance}.Imperial
%College press




%\bibitem{GEF}
%G.E. Fasshauer,\emph{Meshfree Methods}.Handbook of Theoretical and
%Computational Nanotechnology,M.Rieth and W.Schommers(eds.).American
%Scientific Publishers .(2006)
%\end{thebiblio}
\newpage
%=================================Appendix separation page ===========================================================
%\Appendixpage
%---------------------------------------------------------------------------------------------------------------------

%%================================================= Appendix A =========================================================
%\appen{\uppercase{ Sample Appendix}}\label{app1}
%%-------------------------------------------------------------------------------------------------------------------






%=========================Appendix ===================================================================================
%\appen{\uppercase{Computer Programming Codes}}\label{app2}
%----------------------------------------------------------------------------------------------------------------------

\subsection{Computer Programming Codes}
\subsubsection{Tridiagonal solver}
\lstinputlisting{hw1.m}
%\subsubsection{Theta method codes}
%\lstinputlisting{RBasis3GaussianG.m}
%\lstinputlisting{RBasis3GaussianMQ.m}
%\lstinputlisting{RBasis3GaussianIMQ.m}
%\lstinputlisting{tridiagonal.m}




\end{document}
