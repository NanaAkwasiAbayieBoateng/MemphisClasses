
\documentclass[11pt]{article}

\usepackage{ifpdf}
\ifpdf 
    \usepackage[pdftex]{graphicx}   % to include graphics
    \pdfcompresslevel=9 
    \usepackage[pdftex,     % sets up hyperref to use pdftex driver
            plainpages=false,   % allows page i and 1 to exist in the same document
            breaklinks=true,    % link texts can be broken at the end of line
            colorlinks=true,
            pdftitle=My Document
            pdfauthor=My Good Self
           ]{hyperref} 
    \usepackage{thumbpdf}
\else 
    \usepackage{graphicx}       % to include graphics
    \usepackage{hyperref}       % to simplify the use of \href
\fi 

\title{Advanced Macroeconomic II}
\author{Nana Boateng}
\date{}

\begin{document}
\maketitle
\section{Question 2}

1)\\

The Bellman is given as:\\

$v(a,w)=\max  \limits_{a^{\prime} \geq 0} \ [ u(c)+\beta\sum_{j=1}^{N} q_{ij}v(a\prime 
,w)]$\\

s.t. to the budget constraint\\

$(w+a-c)(1+r)=a^{\prime}$\\

state space $s=[0,\cdots ,a_{max}]$\\
                    $x=[w,\cdots,w]$\\
   
   
 The value function$v$ inherits the properties of $u$ such as\\
 
1)  The value function is bounded on s\\
2) It is continuos,strictly increasing,strictly concave and twice 
differentiable.\\
3)Monotonically increasing in a



2)\\
$c=w+a-\frac{a\prime}{1+r}$\\

FOC: w.r.t. $a^{\prime}$\\

$-\frac{1}{1+r} u^{\prime}\left(  w+a-\frac{a^{\prime}}{1+r}\right)+\beta\sum_{j=1}^{N}q_{ij}v_{1}(a^{\prime},w) \geq 
0$\\

 Envelope condition:\\
 
 $v_{a}(a,w)=u^{\prime}\left( w+a-\frac{a^{\prime}}{1+r} \right)$
 
 Euler condition in terms of c:\\
 
 $u^{\prime}(c)=\beta (1+r) \sum_{j=1}^{N}q_{ij}u^{\prime}(c^\prime)$\\
 
 3)\\
 
 If the individual starts  with a high initial wealth endowment $w$above the 
 borrowing constraint zero ,$a^{\prime}>0$,it will never be binding because the endowment 
 will perpetuate itself and smooth out consumption.\\
 In the case of realization of low income $a^{\prime}=0$,the individual has to 
 borrow to finance consumption,but the Zero borrowing constraint becomes 
 binding.\\
 Growth of assets is negatively related to future expected growth of income,If 
 the individual starts above the borrowing constraint
 If the income process is  i.i.d. ,we have $\Delta a_{t}=\epsilon$  which means  that  
 wealth  and a random walk and as a result ,any constraint on asset holdings 
 will be binding with probability of 1.Borrowing becomes binding or not 
 depending on the income process.\\
 
 $u^{\prime}(c)=Eu^{\prime}(c^{\prime}) \hspace{5mm}$ if $a^{\prime}>0$\\
                
   $c_{t}=w_{t}+a \hspace{5mm}$ if $a^{\prime}=0$\\             
                
4)   The optimal policies for consumption and asset accumulation $\gamma(a,\theta)$     and $s(a,\theta)$
    as functions of $a$ increase in $a$         
    The optimal policies for consumption and asset accumulation $\gamma(a,\theta)$     and $s(a,\theta)$
    as functions of $a$ are the optimal policy functions which solves the consumption-savings solves the 
    consumption savings problem.
           
 5)
 If $\theta$ is i.i.d\\
 The bellman becomes 
       $v(a)=\max  \limits_{a^{\prime} \geq 0} \ [ u(c)+\beta\sum_{i=1}^{N} q_{i}v(a\prime 
)]$\\       
              
  the transition matrix is no longer stochastic.The probability of moving from on state to the other does not depend on the
  previous state but random.\\            
              
              
6)

$w_{t}>c_{t}$
                           
                             
  \section{Question 1}              
 
 1)\\
 $v^{u}=\max  \limits_{\pi} \left[ b-c(\pi)+\beta((1-\pi)v^{u}+\pi \int_{0}^{\bar{w}}v^{E}(w)f(w)dw) \right]$  \\
 
 
$0=-c^{\prime}(\pi^{*})+\beta v^{u}+\beta \int_{0}^{\bar{w}}v^{E}(w)f(w)dw-\beta v^{u}$ \\
  
$c^{\prime}(\pi^{*})=\beta v^{u}+\beta \int_{0}^{\bar{w}}v^{E}(w)f(w)dw-\beta v^{u}$\\

$v^{E}(\hat{w}) = \left\{ \begin{array}{rcl}
v^{u}   \mbox{\hspace{5mm}if  $\hat{w} \leq w_{R}$}
& \\ \hat{w} +\beta[\lambda v^{u}+(1-\lambda)v^{E}(w)]& \mbox{if $\hat{w} \geq w_{R}$} &\\

\end{array}\right.$\\

Let $w_{R}$ be the reservation wage,if $w_{R} \leq \hat{w}$\\

$\frac{v^{E}(\hat{w})}{\beta}=\frac{\hat{w}}{\beta}+\lambda 
v^{u}+(1-\lambda)v^{E}(\hat{w})$\\

$\frac{w_R}{\beta}=\frac{v^{E}(\hat{w})}{\beta}-\lambda v^{E}(\hat{w})+v^{E}(\hat{w})+\lambda 
v^{u}$\\

$\frac{w_{R}}{\beta}=\left()\frac{1+\beta}{\beta} 
\right)v^{E}(\hat{w})+\lambda(v^{u}-v^{E})$\\

$w_{R}=v^{E}(\hat{w})(1+\beta)+\lambda\beta(v^{u}-v^{E})$\\

$w_{R}=v^{E}(\hat{w})[1+\beta(1-\lambda)]+\beta\lambda v^{u}$\\

\begin{figure}[ht!]
\centering
\includegraphics[width=120mm,height=100mm]{macro.jpg}
%\caption{Demand function for popcorn\label{overflow}}
\end{figure}                           

2)\\

$\frac{\partial w_{R}}{\partial \beta}=v^{E}(\hat{w})[1+(1-\lambda)]+\lambda v^{u}  
>0$\\



As $\beta$ increases,$c^{\prime}(\pi)$ also increases.$C$ is a strictly convex 
function.$\pi$,$v^{E}(\hat{w})$ and $v^{u}$ increases accordingly.\\

3)\\
$\frac{\partial w_{R}}{\partial \lambda}=-\beta v^{E}(\hat{w})+\beta v^{u}$\\

As the probability of becoming unemployed in the next period $\lambda$ increases ,$v^{E}$ decreases 
 whereas $v^{u}$ increases.A change in $\lambda$ has not directly  affect $\pi^{*}$\\

4) 
$F \succeq_{FOSD} G $ if and only if $F(\hat{w}) $ $\leq G(\hat{w})$ for all $\hat{w}$.F is an 
non decreasing function.
A shift in the wage distribution from $F$ to $G$  causes ,$\pi$,$v^{E}(\hat{w})$ and $v^{u}$ 
all decrease.

5)$G$ is mean preserving spread of $F$ if and only if $y=x+\epsilon$ for some $x\sim 
F$,$y\sim G$ and $\epsilon$ such that $E(\epsilon|x)=0$ for all $x$.
Since $G$ is mean preserving spread$F(\hat{w})=G(\hat{w})$,it does not cause a change in the solution 
of $\pi$,$v^{E}(\hat{w})$ and $v^{u}$.\\

6)\\
$F \succeq_{SOSD} G $ if and only if\\

$\int_{0}^{\bar(x)}F(\hat{w})d\hat{w} \leq \int_{0}^{\bar(x)}G(\hat{w})d\hat{w} $ 
for all $\bar{x} \in [0,\infty]$\\

A shift in the wage distribution from $F$ to $G$  causes ,$\pi$,$v^{E}(\hat{w})$ and $v^{u}$ 
all decrease.









\end{document}  
