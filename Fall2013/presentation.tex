\documentclass[]{beamer}
% Class options include: notes, notesonly, handout, trans,
%                        hidesubsections, shadesubsections,
%                        inrow, blue, red, grey, brown

% Theme for beamer presentation.
\usepackage{beamerthemesplit} 
% Other themes include: beamerthemebars, beamerthemelined, 
%                       beamerthemetree, beamerthemetreebars  
\usepackage{hyperref}
\title{Beamer slide examples}    % Enter your title between curly braces
\author{My name}                 % Enter your name between curly braces
\institute{My organization}      % Enter your institute name between curly braces
\date{\today}                    % Enter the date or \today between curly braces

\begin{document}

% Creates title page of slide show using above information
\begin{frame}
  \titlepage
\end{frame}
\note{Talk for 30 minutes} % Add notes to yourself that will be displayed when
                           % typeset with the notes or notesonly class options

\section[Outline]{}

% Creates table of contents slide incorporating
% all \section and \subsection commands
\begin{frame}
  \tableofcontents
\end{frame}


\section{Simple slide with three points shown all at once}

\begin{frame}
  \frametitle{Simple slide with three points shown all at once}   % Insert frame title between curly braces

  \begin{itemize}
  \item Point 1
  \item Point 2
  \item Point 3
  \end{itemize}
\end{frame}
\note[enumerate]       % Add notes to yourself that will be displayed when
{                      % typeset with the notes or notesonly class options
\item Note for Point 1   
\item Note for Point 2   
}









\begin{frame}
%\begin{figure}[htb]
%\centering
%\includegraphics[width=0.8\textwidth]{densityf.png}
%\includegraphics[width=10cm,height=7cm]{densityf.png}
%\caption{Kernel Density,$N(b,v)$ and $N(0,V_l)$}
%\label{fig:awesome_image}
%\end{figure}

\end{frame}




\begin{frame}
%\begin{figure}[htb]
%\centering
%\includegraphics[width=10cm,height=7cm]{averagef.png}
%\caption{Scatter plots of  Sample Standard Deviation versus Sample Average generated byNon-Parametric Bootstrap}
%\label{fig:awesome_image}
%\end{figure}

\end{frame}

\begin{frame}
%\begin{figure}[htb]
%\centering
%\includegraphics[width=10cm,height=7cm]{histf.png}
%\caption{Histogram plots of Baeline$( top)$ and oneyear$down$}
%\label{fig:awesome_image}
%\end{figure}

\end{frame}

\begin{frame}
%\begin{figure}[htb]
%\centering
%\includegraphics[width=10cm,height=7cm]{quantilebaseline.png}
%\caption{Quantile plots of CD4 at Baseline year}
%\label{fig:awesome_image}
%\end{figure}

\end{frame}


\begin{frame}
%\begin{figure}[htb]
%\centering
%\includegraphics[width=10cm,height=7cm  ]{quantileoneyear.png}
%\caption{Quantile plots of CD4 After oneyear}
%\label{fig:awesome_image}
%\end{figure}

\end{frame}

\subsection{Simple slide with three points shown in succession}

\begin{frame}
  \frametitle{Simple slide with three points shown in succession}   % Insert frame title between curly braces

  \begin{itemize}
  \item<1-> Point 1 (Click ``Next Page'' to see Point 2) % Use Next Page to go to Point 2
  \item<2-> Point 2  % Use Next Page to go to Point 3
  \item<3-> Point 3
  \end{itemize}
\end{frame}
\note{Speak clearly}  % Add notes to yourself that will be displayed when
                      % typeset with the notes or notesonly class options


\section{Slide with two columns: items and a graphic}

\begin{frame}
  \frametitle{Slide with two columns: items and a graphic}   % Insert frame title between curly braces
  \begin{columns}[c]
  \column{2in}  % slides are 3in high by 5in wide
  \begin{itemize}
  \item<1-> First item
  \item<2-> Second item
  \item<3-> ...
  \end{itemize}
  \column{2in}
  \framebox{Insert graphic here % e.g. \includegraphics[height=2.65in]{graphic}
  }
  \end{columns}
\end{frame}
\note{The end}       % Add notes to yourself that will be displayed when
		     % typeset with the notes or notesonly class options

\end{document}
