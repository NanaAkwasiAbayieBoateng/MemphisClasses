\documentclass[hyperref={pdfpagelabels=false}]{beamer}
% By  using hyperref={pdfpagelabels=false} you get rid off:
% Package hyperref Warning: Option `pdfpagelabels' is turned off
% (hyperref)                because \thepage is undefined. 
% Hyperref stopped early 
%

\usepackage{lmodern}
% Using lmondern and you get rid off this:
% LaTeX Font Warning: Font shape `OT1/cmss/m/n' in size <4> not available
% (Font)              size <5> substituted on input line 22.
% LaTeX Font Warning: Size substitutions with differences
% (Font)              up to 1.0pt have occurred.
%

% If \titel{$B!D(B} \author{$B!D(B} come after \begin{document} 
% you get the following warnig:
% Package hyperref Warning: Option `pdfauthor' has already been used,
% (hyperref) ... 
% So it is here before \begin{document}

\usepackage[T1]{fontenc}

\usepackage{fix-cm}

\title{Understanding The Metropolis-Hastings Algorithm}   
\author{Nana Boateng 
} 
\date{\today} 
% additional usepackage{beamerthemeshadow} is used

\usepackage{beamerthemeshadow}
%\usetheme{CambridgeUS}
    \usepackage[latin1]{inputenc}
    \usefonttheme{professionalfonts}
    \usepackage{times}
    \usepackage{tikz}
    \usepackage{amsmath}
    \usepackage{verbatim}
    \usepackage[T1]{fontenc}
    \usepackage{mathtools}
    \usepackage{fix-cm}
    \usetikzlibrary{arrows,shapes}
    \DeclareMathOperator*{\Max}{Max}

\usepackage{graphicx}% http;//ctan.org/pkg/graphicx
\usepackage{amsfonts}
\usepackage{amssymb}
\usepackage{amsmath}
\usepackage{fancyvrb}
\usepackage{listings}
\usepackage{lstautogobble}

\lstset{basicstyle=\ttfamily,
  mathescape=true,
  escapeinside=||,
  autogobble}
% Let us first define a custom Verbatim environment, that saves us a lot of writing
  \DefineVerbatimEnvironment{VerbatimTest}{Verbatim}%
    {showspaces,showtabs,commandchars=\\\{\}}% showspaces and showtabs only for visualizing
\lstdefinestyle{customc}{
  belowcaptionskip=1\baselineskip,
  breaklines=true,
  frame=L,
  xleftmargin=\parindent,
  language=SAS,
  showstringspaces=false,
  basicstyle=\footnotesize\ttfamily,
  keywordstyle=\bfseries\color{green!40!black},
  commentstyle=\itshape\color{purple!40!black},
  identifierstyle=\color{blue},
  stringstyle=\color{orange},
}

\lstdefinestyle{customasm}{
  belowcaptionskip=1\baselineskip,
  frame=L,
  xleftmargin=\parindent,
  language=[x86masm]Assembler,
  basicstyle=\footnotesize\ttfamily,
  commentstyle=\itshape\color{purple!40!black},
}

\lstset{escapechar=@,style=customc}
\lstset{language=SAS,caption={Descriptive Caption Text},label=DescriptiveLabel}
\usepackage{listings}
\usepackage{color}

\definecolor{mygreen}{rgb}{0,0.6,0}
\definecolor{mygray}{rgb}{0.5,0.5,0.5}
\definecolor{mymauve}{rgb}{0.58,0,0.82}

\lstset{ %
  backgroundcolor=\color{white},   % choose the background color; you must add \usepackage{color} or \usepackage{xcolor}
  basicstyle=\footnotesize,        % the size of the fonts that are used for the code
  breakatwhitespace=false,         % sets if automatic breaks should only happen at whitespace
  breaklines=true,                 % sets automatic line breaking
  captionpos=b,                    % sets the caption-position to bottom
  commentstyle=\color{mygreen},    % comment style
  deletekeywords={...},            % if you want to delete keywords from the given language
  escapeinside={\%*}{*)},          % if you want to add LaTeX within your code
  extendedchars=true,              % lets you use non-ASCII characters; for 8-bits encodings only, does not work with UTF-8
  frame=single,                    % adds a frame around the code
  keepspaces=true,                 % keeps spaces in text, useful for keeping indentation of code (possibly needs columns=flexible)
  keywordstyle=\color{blue},       % keyword style
  language=Octave,                 % the language of the code
  morekeywords={*,...},            % if you want to add more keywords to the set
  numbers=left,                    % where to put the line-numbers; possible values are (none, left, right)
  numbersep=5pt,                   % how far the line-numbers are from the code
  numberstyle=\tiny\color{mygray}, % the style that is used for the line-numbers
  rulecolor=\color{black},         % if not set, the frame-color may be changed on line-breaks within not-black text (e.g. comments (green here))
  showspaces=false,                % show spaces everywhere adding particular underscores; it overrides 'showstringspaces'
  showstringspaces=false,          % underline spaces within strings only
  showtabs=false,                  % show tabs within strings adding particular underscores
  stepnumber=2,                    % the step between two line-numbers. If it's 1, each line will be numbered
  stringstyle=\color{mymauve},     % string literal style
  tabsize=2,                       % sets default tabsize to 2 spaces
  title=\lstname                   % show the filename of files included with \lstinputlisting; also try caption instead of title
}


\usepackage{tabularx}
\usepackage{multirow}
\usepackage{color}





%\titleformat{\section}[block]{\color{black}\Large\bfseries\filcenter}{}{1em}{}
%\titleformat{\subsection}[hang]{\bfseries}{}{1em}{}

\setcounter{secnumdepth}{1}

% allows use of "@" in control sequence names
\makeatletter
\DeclareGraphicsExtensions{.pdf,.png,.jpg}
% this creates a custom and simpler ruled box style
\newcommand\floatc@simplerule[2]{{\@fs@cfont #1 #2}\par}
\newcommand\fs@simplerule{\def\@fs@cfont{\bfseries}\let\@fs@capt\floatc@simplerule
  \def\@fs@pre{\hrule height.8pt depth0pt \kern4pt}%
  \def\@fs@post{\kern4pt\hrule height.8pt depth0pt \kern4pt \relax}%
  \def\@fs@mid{\kern8pt}%
  \let\@fs@iftopcapt\iftrue}





\def\aa{\vec{a}}
\def\bb{\vec{b}}
\def\cc{\vec{c}}
\def\ee{\vec{e}}
\def\pp{\vec{p}}
\def\xx{\vec{x}}
\def\yy{\vec{y}}
\def\zz{\vec{z}}
\def\PP{\vec{P}}
\def\SS{\vec{S}}
\def\aalpha{\vec{\alpha}}
\def\bbeta{\vec{\beta}}
\def\ggamma{\vec{\gamma}}
\def\llambda{\vec{\lambda}}
\def\nnu{\vec{\nu}}

\def\cD{\mathcal{D}}
\def\cF{\mathcal{F}}
\def\cG{\mathcal{G}}
\def\cI{\mathcal{I}}
\def\cL{\mathcal{L}}
\def\cM{\mathcal{M}}
\def\cN{\mathcal{N}}
\def\cO{\mathcal{O}}
\def\cP{\mathcal{P}}
\def\cQ{\mathcal{Q}}
\def\cS{\mathcal{S}}
\def\cX{\mathcal{X}}
\DeclareMathAlphabet{\mathpzc}{OT1}{pzc}{m}{it}

%\usepackage{statrep}
\begin{document}

\begin{frame}
\titlepage
\end{frame} 

\begin{frame}
\frametitle{Table of contents}
\tableofcontents
\end{frame}
 


\section{Introduction} 

%\begin{frame}
%\frametitle{Background }
%\begin{itemize}
%\item Developed by Metropolis,Rosenbluth,Rosenbluth,Teller and Teller(1953)
%\item  Developed by Nicholas Metropolis, Arianna W. Rosenbluth, Marshall N. Rosenbluth, Augusta H. Teller, and Edward %Teller(1953)
%\item Generalized by Hastings in 1970
%\item The main idea behind the Metropolis-Hastings algorithm  is to simulate an egordic Markov chain such that the  stationary %distribution  of this chain coincides with the target distribution
%\end{itemize}
%\end{frame}

%\begin{frame}
%\begin{itemize}
%\item Assume we want to generate a random variable $X$ taking values $\mathcal{X}=\{1,\cdots,m\}$ according to a target %distribution $\{\pi_{i}\}$,with \\
%\vspace{10mm}
$\pi_{i}=\frac{b_{i}}{C}$, \hspace{10mm}$i \in \mathcal{X}$
%\end{frame}

%\end {itemize}
%where it is assumed that all $\{b_{i}\}$ are strictly positive,$m$ is large,and the normalization constant 
%$C=\sum_{i=1}^m b_{i}$ is difficult to calculate.
%\end{frame}
\end{document}
