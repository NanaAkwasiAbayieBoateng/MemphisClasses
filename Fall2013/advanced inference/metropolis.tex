\documentclass[hyperref={pdfpagelabels=false}]{beamer}
% By  using hyperref={pdfpagelabels=false} you get rid off:
% Package hyperref Warning: Option `pdfpagelabels' is turned off
% (hyperref)                because \thepage is undefined. 
% Hyperref stopped early 
%

\usepackage{lmodern}
% Using lmondern and you get rid off this:
% LaTeX Font Warning: Font shape `OT1/cmss/m/n' in size <4> not available
% (Font)              size <5> substituted on input line 22.
% LaTeX Font Warning: Size substitutions with differences
% (Font)              up to 1.0pt have occurred.
%

% If \titel{$B!D(B} \author{$B!D(B} come after \begin{document} 
% you get the following warnig:
% Package hyperref Warning: Option `pdfauthor' has already been used,
% (hyperref) ... 
% So it is here before \begin{document}

\usepackage[T1]{fontenc}

\usepackage{fix-cm}

\title{Understanding The Metropolis-Hastings Algorithm}   
\author{Nana Boateng } 
\date{\today} 
% additional usepackage{beamerthemeshadow} is used

\usepackage{beamerthemeshadow}
%\usetheme{CambridgeUS}
    \usepackage[latin1]{inputenc}
    \usefonttheme{professionalfonts}
    \usepackage{times}
    \usepackage{tikz}
    \usepackage{amsmath}
    \usepackage{verbatim}
    \usepackage[T1]{fontenc}
    \usepackage{mathtools}
    \usepackage{fix-cm}
    \usetikzlibrary{arrows,shapes}
    \DeclareMathOperator*{\Max}{Max}

\usepackage{graphicx}% http;//ctan.org/pkg/graphicx
\usepackage{amsfonts}
\usepackage{amssymb}
\usepackage{amsmath}
\usepackage{fancyvrb}
\usepackage{listings}
\usepackage{lstautogobble}

\lstset{basicstyle=\ttfamily,
  mathescape=true,
  escapeinside=||,
  autogobble}
% Let us first define a custom Verbatim environment, that saves us a lot of writing
  \DefineVerbatimEnvironment{VerbatimTest}{Verbatim}%
    {showspaces,showtabs,commandchars=\\\{\}}% showspaces and showtabs only for visualizing
\lstdefinestyle{customc}{
  belowcaptionskip=1\baselineskip,
  breaklines=true,
  frame=L,
  xleftmargin=\parindent,
  language=SAS,
  showstringspaces=false,
  basicstyle=\footnotesize\ttfamily,
  keywordstyle=\bfseries\color{green!40!black},
  commentstyle=\itshape\color{purple!40!black},
  identifierstyle=\color{blue},
  stringstyle=\color{orange},
}

\lstdefinestyle{customasm}{
  belowcaptionskip=1\baselineskip,
  frame=L,
  xleftmargin=\parindent,
  language=[x86masm]Assembler,
  basicstyle=\footnotesize\ttfamily,
  commentstyle=\itshape\color{purple!40!black},
}

\lstset{escapechar=@,style=customc}
\lstset{language=SAS,caption={Descriptive Caption Text},label=DescriptiveLabel}
\usepackage{listings}
\usepackage{color}

\definecolor{mygreen}{rgb}{0,0.6,0}
\definecolor{mygray}{rgb}{0.5,0.5,0.5}
\definecolor{mymauve}{rgb}{0.58,0,0.82}

\lstset{ %
  backgroundcolor=\color{white},   % choose the background color; you must add \usepackage{color} or \usepackage{xcolor}
  basicstyle=\footnotesize,        % the size of the fonts that are used for the code
  breakatwhitespace=false,         % sets if automatic breaks should only happen at whitespace
  breaklines=true,                 % sets automatic line breaking
  captionpos=b,                    % sets the caption-position to bottom
  commentstyle=\color{mygreen},    % comment style
  deletekeywords={...},            % if you want to delete keywords from the given language
  escapeinside={\%*}{*)},          % if you want to add LaTeX within your code
  extendedchars=true,              % lets you use non-ASCII characters; for 8-bits encodings only, does not work with UTF-8
  frame=single,                    % adds a frame around the code
  keepspaces=true,                 % keeps spaces in text, useful for keeping indentation of code (possibly needs columns=flexible)
  keywordstyle=\color{blue},       % keyword style
  language=Octave,                 % the language of the code
  morekeywords={*,...},            % if you want to add more keywords to the set
  numbers=left,                    % where to put the line-numbers; possible values are (none, left, right)
  numbersep=5pt,                   % how far the line-numbers are from the code
  numberstyle=\tiny\color{mygray}, % the style that is used for the line-numbers
  rulecolor=\color{black},         % if not set, the frame-color may be changed on line-breaks within not-black text (e.g. comments (green here))
  showspaces=false,                % show spaces everywhere adding particular underscores; it overrides 'showstringspaces'
  showstringspaces=false,          % underline spaces within strings only
  showtabs=false,                  % show tabs within strings adding particular underscores
  stepnumber=2,                    % the step between two line-numbers. If it's 1, each line will be numbered
  stringstyle=\color{mymauve},     % string literal style
  tabsize=2,                       % sets default tabsize to 2 spaces
  title=\lstname                   % show the filename of files included with \lstinputlisting; also try caption instead of title
}


\usepackage{tabularx}
\usepackage{multirow}
\usepackage{color}





%\titleformat{\section}[block]{\color{black}\Large\bfseries\filcenter}{}{1em}{}
%\titleformat{\subsection}[hang]{\bfseries}{}{1em}{}

\setcounter{secnumdepth}{1}

% allows use of "@" in control sequence names
\makeatletter
\DeclareGraphicsExtensions{.pdf,.png,.jpg}
% this creates a custom and simpler ruled box style
\newcommand\floatc@simplerule[2]{{\@fs@cfont #1 #2}\par}
\newcommand\fs@simplerule{\def\@fs@cfont{\bfseries}\let\@fs@capt\floatc@simplerule
  \def\@fs@pre{\hrule height.8pt depth0pt \kern4pt}%
  \def\@fs@post{\kern4pt\hrule height.8pt depth0pt \kern4pt \relax}%
  \def\@fs@mid{\kern8pt}%
  \let\@fs@iftopcapt\iftrue}





\def\aa{\vec{a}}
\def\bb{\vec{b}}
\def\cc{\vec{c}}
\def\ee{\vec{e}}
\def\pp{\vec{p}}
\def\xx{\vec{x}}
\def\yy{\vec{y}}
\def\zz{\vec{z}}
\def\PP{\vec{P}}
\def\SS{\vec{S}}
\def\aalpha{\vec{\alpha}}
\def\bbeta{\vec{\beta}}
\def\ggamma{\vec{\gamma}}
\def\llambda{\vec{\lambda}}
\def\nnu{\vec{\nu}}

\def\cD{\mathcal{D}}
\def\cF{\mathcal{F}}
\def\cG{\mathcal{G}}
\def\cI{\mathcal{I}}
\def\cL{\mathcal{L}}
\def\cM{\mathcal{M}}
\def\cN{\mathcal{N}}
\def\cO{\mathcal{O}}
\def\cP{\mathcal{P}}
\def\cQ{\mathcal{Q}}
\def\cS{\mathcal{S}}
\def\cX{\mathcal{X}}
%\setbeamertemplate{footline}{\insertframenumber/\inserttotalframenumber}
%\useoutertheme{infolines}
%\usetheme{Hannover}
%\setcounter{captureframe}{\value{framenumber}}
\mode<presentation>
 {
   \usetheme{Dresden}
 }

 \setbeamertemplate
 {footline}{\quad\hfill\insertframenumber/\inserttotalframenumber\strut\quad} 
\everymath{\displaystyle} 


\begin{document}

\begin{frame}
\titlepage
\end{frame} 

\begin{frame}
\frametitle{Table of contents}
\tableofcontents
\end{frame}
 



\section{Introduction} 

\begin{frame}
\frametitle{Background }
\begin{itemize}
%\item Developed by Metropolis,Rosenbluth,Rosenbluth,Teller and Teller(1953)
\item  Developed by Nicholas Metropolis, Arianna W. Rosenbluth, Marshall N. Rosenbluth, Augusta H. Teller, and Edward Teller(1953)
\item Generalized by Hastings in 1970
\item The main idea behind the Metropolis-Hastings algorithm  is to simulate an egordic Markov chain such that the  stationary distribution  of this chain coincides with the target distribution
\end{itemize}
\end{frame}


\begin{frame}
\begin{itemize}
\item Assume we want to generate a random variable $X$ taking values $\mathcal{X}=\{1,\cdots,m\}$ according to a target distribution $\{\pi_{i}\}$,with \\
\vspace{5mm}
$\pi_{i}=\frac{b_{i}}{C}$, \hspace{10mm}$i \in \mathcal{X}$
%\end{frame}

\end{itemize}
where it is assumed that all $\{b_{i}\}$ are strictly positive,$m$ is large,and the normalization constant 
$C=\sum_{i=1}^m b_{i}$ is difficult to calculate.

\end{frame}

\section{Acceptance-Rejection Sampling}
\begin{frame}
\frametitle{Acceptance-Rejection Sampling}
The objective is to generate samples from the  absolutely continous \textit{target density} $\pi(x)=\frac{f(x)}{K}$\\
Let $h(x)$ be a density that can be simulated by some known method and suppose there is a constant $c$ such that
$f(x)\leq ch(x)$
\begin{itemize}
\item $(*)$ Generate a candidate Z from $h(.)$ and a value $u$ from $U(0,1)$,the uniform distribution on $(0,1)$
\item $u\leq \frac{f(Z)}{ch(Z)}$
\begin{itemize}
\item return  $Z=y$
\end{itemize}
\item Else\\
\begin{itemize}
\item  go to $(*)$
\end{itemize}

\end{itemize}
\end{frame}

\begin{frame}
For this method to be efficient,$c$ must be carefully selected .Because the expected number of iterations of steps $1$ and $2$  to obtain a draw is given by $c^{-1}$,the rejection method is optimized by setting 

$
c=\sup_{x}\frac{f(x)}{h(x)}
$

\end{frame}

\section{The Metroplois Hastings Algorithm}

%\begin{frame}
%\frametitle{The Metroplois Hastings Algorithm}

%\end{frame}
\begin{frame}
\frametitle{The Metroplois Hastings Algorithm}
%\pause
\begin{itemize}

\pause
\item We denote the candidate generating density $q(x,y)$,where $\int q(x,y)=1$
\pause
\item Interpreted as saying when the proceess is at the point $x$,the density generates a value $y$ from $q(x,y)$

\pause
\item We want $q(x,y)$ to satisfy the reversibility condition $\pi(x)q(x,y)=\pi(y)q(y,x)$.Thus the process moves from $x$ to $y$ equal number of times.

\pause 
\item Suppose that  $\pi(x)q(x,y)>\pi(y)q(y,x)$ $*$,The process moves from $x$ to $y$ too often and from $y$ to $x$ rarely.

\pause
\item A convenient  way of correcting this condition is by introducing a probability $\alpha(x,y)<1$ referred to as probability of move.
\end{itemize}
\end{frame}

\begin{frame}
\frametitle{The Metroplois Hastings Algorithm}
\begin{itemize}
\pause
\item The transitions from $x$ to $y$ are made according to $\rho_{MH}=q(x,y)\alpha(x,y)$
\pause
\item $\alpha(x,y) $ is determined by requiring $\rho_{MH}$ to satisfy the reversibility condition.then
$\pi(x)q(x,y)\alpha(x,y)=\pi(y)q(y,x)\alpha(y,x)=\pi(y)q(y,x)$

\vspace{5mm}

$\alpha(x,y)=\frac{\pi(y)q(y,x)}{\pi(x)q(x,y)}$

\pause
\item We can reverse  $*$ and set $\alpha(x,y)=1$ and derive $\alpha(y,x)$ similarly.
\end{itemize}
\end{frame}

\begin{frame}
\frametitle{The Metroplois Hastings Algorithm}
\begin{itemize}

\pause
\item In order for $\rho_{MH}$ to be reversible we set

%\begin{equation*}

\[ \alpha(x,y) = \left\{ 
 \begin{array}{l l}
 \textrm{min}\Bigg[ \frac{\pi(y)q(y,x)}{\pi(x)q(x,y)},1\Bigg] & \quad \textrm{,if $\pi(y)q(y,x)>0$}\\
   1 & \quad \textrm{,otherwise}
  \end{array} \right.\]
\pause
\item the non-zero probability that the process remains at $x$  is defined  by $ r(x)=1-\int_{R^{d}}q(x,y)\alpha(x,y)dy$


\end{itemize}

\end{frame}

\begin{frame}
\begin{itemize}
\pause
\item consequently the transition kernel of the M-H chain 
%\begin{equation}
$\rho_{MH  }(x,dy)$ is defined by\\
 \vspace{3mm}
$\rho_{MH  }(x,dy)=q(x,y)\alpha(x,y)dy+\Bigg[1-\int_{R^{d}}q(x,y)\alpha(x,y)\delta_{x}(dy)\Bigg]$
%\end{equation}

\pause
\item  Since $\rho_{MH  }(x,y)$ is reversible by construction ,it follows that M-H kernel has $\pi(x)$ as its invariant density.
\pause
\item If the candidate generating density is symmetric thus $q(x,y)=q(y,x)$ then \\
$\alpha(x,y)=\frac{\pi(y)}{\pi(x)}$
\end{itemize}
\end{frame}


\begin{frame}

\begin{figure}[h!]
  
  \centering
    \includegraphics[width=10cm,height=5cm]{met1.png}
\caption{Calculating probabilities of move with symmetric Candidate-Generating Function}
\label{fig:awesome_image}
\end{figure}
\end{frame}


\begin{frame}
\frametitle{Algorithm}
\begin{itemize}
\pause
\item Iniatialize with arbitrary value $x^{(0)}$
\pause
\item Repeat for $j=1,2\cdots N$
\pause
\item Generate y from $q(x^{(j)},.)$ and $u$ from $U(0,1)$

\pause
\item $u\leq \alpha(x^{(j)},y)$,set $x^{(j+1)}=y$ 
\item Else set $x^{(j+1)}=x^{(j)}$ 
\pause
\item Return values ${x^{(1)},x^{(2)},\cdots,x^{(N)}}$
\vspace{5mm}
\item The regularity condition required for the draws to be regarded as a sample from the target density is irreducibility and aperiodicity
\end{itemize}
\end{frame}


\section{Implementation Issues:Choice of $q(x,y)$}

\begin{frame}
\frametitle{Implementation Issues:Choice of $q(x,y)$}
\begin{itemize}
\item Random Walk\\
$q(x,y)=q_{1}(y-x)$

The proposal sate $y$ for a given current state $x$,is given by $y=x+z$ where $z$ is the increment  $\sim q_{1}$\\
The possible choice of $q_{1}$  include Multivariate normal and t -distributions,
$\alpha(x,y)=\textrm{min}\Bigg[\frac{\pi(y)}{\pi(x)},1\Bigg] $

\pause
\item Independence Sampler
$q(x,y)$ is chosen to be independent to $x$ ie $q(x,y)=q_{2}(y)$

$q_{2}(y) \sim  N_{m}(\mu,\Sigma)$ and multivariate t

\end{itemize}

\end{frame}


\begin{frame}
\frametitle{Implementation Issues:Choice of $q(x,y)$}
\begin{itemize}
\item third method specifies the candiate- generating density Eg. $\pi(t)=\psi(t)h(t)$
where $h(t)$ is a density that can be sampled and $\psi(t)$ is uniformly  bounded then set\\
 $q(x,y)=h(y)$\\
$\alpha(x,y)=\textrm{min}\Bigg[\frac{\psi(y)}{\psi(x)},1\Bigg] $

\pause
\item Fourth method is to use A-R method with pseudo dominating density


\pause
\item Fifth method is represented by a vector of autoregressive process of order 1\\
$y=a +B(x-a)+z$,$z \sim q$ density
Then $q(x,y)=q(y-a-B(x-a)$
\end{itemize}
\end{frame}


\begin{frame}
\frametitle{Applications of The M-H Algorithm}
\begin{itemize}
\item An M-H  Acceptance-Rejection Algorithm
$\pi(x)=\frac{f(x)}{K}$
$f(x)\leq ch(x)$

\end{itemize}
\end{frame}



\begin{frame}
\frametitle{Applications of The M-H Algorithm}
\begin{itemize}
\item Block-at-a-Time Algorithm
Another interesting situation arises when the M-H algorithm is applied in turn to subblocks of the
vector $x$,rather than simultaneously to all elements of the vector.Hastings(1970) discussed in his paper that block-at-a -time algorithm 
 can often simplify  the search for a suitable generating density and gives rise to several intersting hybrid algorithms
obtained by combining M-H updates

\end{itemize}

\end{frame}

\begin{frame}
The central idea behind these algorithms may be illustrated with two blocks,$x=(x_{1},x_{2})$,where $x_{1} \in  \Re^{d}$.
Suppose that there exist a conditional transition kernel $p_{1}(x_{1},dy_{1} \mid x_{2})$ with the property that,
for a fixed value of $x_{2}$,$\pi_{1 \mid 2}^{*}(.\mid x_{2})$ is its invariant distribution with density $\pi_{1 \mid 2}(.\mid x_{2})$,
that is


$\pi_{1 \mid 2}^{*}(dy_{1}\mid x_{2})=\int p_{1}(x_{1},dy_{1}\mid x_{2})\pi_{1 \mid 2}(x_{1}\mid x_{2})dx_{1}$
%$p_{1}(x_{1},dy_{1}\mid x_{2})=\pi _{1 \mid 2}^{*}(dy_{1} \mid x_{2})$
\end{frame}


\begin{frame}
The kernels formed by multiplying the conditional kernels has $\pi^{*}(.,.)$ as its invariant distribution
$\int \int p_{1}(x_{1},dy_{1}\mid x_{2})p_{2}(x_{2},dy_{2}\mid y_{1})\pi(x_{1},x_{2})dx_{1}dx_{2}$
$=\int p_{2}(x_{2},dy_{2}\mid y_{1})\bigg[   \int p_{1}(x_{1},dy_{1}\mid x_{2}) \pi_{1 \mid 2}(x_{1}\mid x_{2})dx_{1} \bigg] \pi_{2}(x_{2})dx_{2}$

$=\int p_{2}(x_{2},dy_{2}\mid y_{1})\pi_{1 \mid 2}^{*}(.\mid x_{2})\pi_{2}(x_{2})dx_{2}$

$=\int p_{2}(x_{2},dy_{2}\mid y_{1})\frac{\pi_{2 \mid 1}(x_{2}\mid y_{1})\pi_{1 }^{*}(dy_{1})}{\pi_{2}(x_{2})}\pi_{2}(x_{2})dx_{2}$


$= \pi_{1}^{*}(dy_{1}) \int p_{2}(x_{2},dy_{2}\mid y_{1}) \pi_{2 \mid 1}(x_{2}\mid y_{1})dx_{2}$


$=\pi^{*}(dy_{1})\pi_{2 \mid 1}^{*}(dy_{2} \mid y_{1})$

$=\pi^{*}(dy_{1},dy_{2})$

\end{frame}



\begin{frame}
\begin{itemize}

\item Gibbs Sampler
The algorithm can be obtained by letting the transition  kernel\\
 %$ P_{1}(x_{1},dy_{1} \mid x_{2})= \pi_{1\mid2}^{*}(dy_{1}\midx_{2}) $
%and $ P_{2}(x_{2},dy_{2}\midy_{})=\pi_{2\mid1}^{*}(dy_{2}\midy_{1}) $
$p_{1}(x_{1},dy_{1}\mid x_{2})=\pi _{1 \mid 2}^{*}(dy_{1} \mid x_{2})$\\

and



$p_{2}(x_{2},dy_{2}\mid y_{1})=\pi _{2 \mid 1}^{*}(dy_{2} \mid y_{1})$ \\
\vspace{3mm}

The case in which all full conditional  kernels are sampled by independent algorithms
in a fixed order

all full conditional kernels 
\end{itemize}

\end{frame}


%\section{}
\begin{frame}
Thank you for your attention
\end{frame}
\end{document}
