\documentclass[11pt]{article}
\usepackage{amsfonts}
\usepackage{amsmath}
\usepackage{color}
\usepackage{tabularx}
\usepackage{multirow}
\usepackage{color}
\usepackage[pdftex]{graphicx}
\usepackage{caption}
\usepackage{chngcntr}
\counterwithin{table}{section}
\usepackage{float}
\floatstyle{boxed}
\restylefloat{figure}
\usepackage[english]{babel}
\usepackage[pdftex]{hyperref}
\usepackage{xcolor}
\usepackage{titlesec}
\usepackage{lipsum}
\usepackage{lmodern}
\usepackage[T1]{fontenc}
\usepackage{textcomp}
\usepackage{fixltx2e}


%\titleformat{\section}[block]{\color{black}\Large\bfseries\filcenter}{}{1em}{}
%\titleformat{\subsection}[hang]{\bfseries}{}{1em}{}

\setcounter{secnumdepth}{1}

% allows use of "@" in control sequence names
\makeatletter
\DeclareGraphicsExtensions{.pdf,.png,.jpg}
% this creates a custom and simpler ruled box style
\newcommand\floatc@simplerule[2]{{\@fs@cfont #1 #2}\par}
\newcommand\fs@simplerule{\def\@fs@cfont{\bfseries}\let\@fs@capt\floatc@simplerule
  \def\@fs@pre{\hrule height.8pt depth0pt \kern4pt}%
  \def\@fs@post{\kern4pt\hrule height.8pt depth0pt \kern4pt \relax}%
  \def\@fs@mid{\kern8pt}%
  \let\@fs@iftopcapt\iftrue}

% this code block defines the new and custom floatbox float environment
\floatstyle{simplerule}
\newfloat{floatbox}{thp}{lob}[section]
\floatname{floatbox}{Text Box}

\usepackage{dcolumn}
%here we're setting up a version of the math fonts with normal x-width
\DeclareMathVersion{nxbold}
\SetSymbolFont{operators}{nxbold}{OT1}{cmr} {b}{n}
\SetSymbolFont{letters}  {nxbold}{OML}{cmm} {b}{it}
\SetSymbolFont{symbols}  {nxbold}{OMS}{cmsy}{b}{n}

\oddsidemargin .27in \evensidemargin .3in \textwidth 6.5in
\topmargin -.25in \textheight 8.75in

\newtheorem{thm}{Theorem}[section]
\newtheorem{cor}[thm]{Corollary}
\newtheorem{lem}[thm]{Lemma}
\newtheorem{prop}[thm]{Proposition}
\newtheorem{defn}[thm]{Definition}
\newtheorem{rem}[thm]{Remark}
\numberwithin{figure}{section}
\renewcommand{\thefootnote}{\fnsymbol{footnote}}

\def\RR{\mathbb{R}}
\def\NN{\mathbb{N}}
\def\ZZ{\mathbb{Z}}
\def\CC{\mathbb{C}}
\def\st{\mathop{\: | \:}\nolimits}

\def\vec#1{\mathchoice{\mbox{\boldmath$\displaystyle#1$}}
{\mbox{\boldmath$\textstyle#1$}}
{\mbox{\boldmath$\scriptstyle#1$}}
{\mbox{\boldmath$\scriptscriptstyle#1$}}}

\def\aa{\vec{a}}
\def\bb{\vec{b}}
\def\cc{\vec{c}}
\def\ee{\vec{e}}
\def\pp{\vec{p}}
\def\xx{\vec{x}}
\def\yy{\vec{y}}
\def\zz{\vec{z}}
\def\PP{\vec{P}}
\def\SS{\vec{S}}
\def\aalpha{\vec{\alpha}}
\def\bbeta{\vec{\beta}}
\def\ggamma{\vec{\gamma}}
\def\llambda{\vec{\lambda}}
\def\nnu{\vec{\nu}}

\def\cD{\mathcal{D}}
\def\cF{\mathcal{F}}
\def\cG{\mathcal{G}}
\def\cI{\mathcal{I}}
\def\cL{\mathcal{L}}
\def\cM{\mathcal{M}}
\def\cN{\mathcal{N}}
\def\cO{\mathcal{O}}
\def\cP{\mathcal{P}}
\def\cQ{\mathcal{Q}}
\def\cS{\mathcal{S}}
\def\cX{\mathcal{X}}

\usepackage{listings}
 \lstloadlanguages{R}

\title{ Comparison of Three Diets}
\author{Nana.~Boateng\footnotemark[1], Will .Terry\footnotemark[2], Jiang\footnotemark[3] and Jingnan
Zhao\footnotemark[7]}

\date{\today}

\begin{document}
\maketitle \thispagestyle{empty}
\begin{abstract}

We compare the survival function of rats in three diet groups, low fat, saturated and unsaturated fat. The K-Sample Gehan$'$s Generalized Wilcoxon Test, Generalized Log rank and Peto-Peto Test are   implemented in testing for significant difference between the three groups. The adjusted significance and p-values of multiple paired comparison test are obtained by the Tukey-Kramer method. We compare the output by the SAS system with that of the R software to determine   its consistency.

\end{abstract}

\fnsymbol{footnote}

\footnotetext[1]{Department of  Mathematics, University of Memphis}
\footnotetext[2]{Department of  Mathematics, University of Memphis}

\section{Introduction}
%Survival analysis examines and models the time it takes for events to occur.A typical event waiting to occur is death.In most %cases %it focuses on techniques  and   distibutionof survival times   forr positive random variables such as
%\begin{itemize}
%\item time to death
%\item time to onset of remission of a disease
%\item duration of hospital stay
%\item viral load measurements
%\end{itemize}
%There are various kinds of survival studies which include clinical trials,prospective and retrospective cohort  studies and %retrospective 
%correlative studies
%\section{Analysis}

%\section{Cox proportional-hazards regression model}
%Cox proportional-hazards regression model examines the relationship
%between survival and one or more predictors, usually termed covariates


%\begin{equation}
%P(t) = Pr(T \leq t)
%\end{equation}
%The Cox model, in contrast, leaves the baseline hazard function α $(t) = log h_{0}(t)$ unspecified:
%\begin{equation}
%log h_{i}(t) =\alpha α(t) +\beta _{1}x_{i1} + \beta_{2}x_{ik} + · · · + \beta_{k}x_{ik}
%\end{equation}
%\begin{thebibliography}{100}


In this study we are interested in investigating relationship between diet and development of tumors. About 90 rats were divided into three groups and fed with low-fat, saturated and unsaturated fat diets. The rats used in the study were of the same age and species and similar physical condition. An identical amount tumor cells were injected into a foot pad of each rat and the rats observed for 200 days.
The data for our study are listed below in the following table. The table below gives the tumor-free time, the time from injection to the time that a tumor develops or to the end of the study.
Our main interest here is to compare the three diets abilities to keep the rats tumor-free. We will estimate the survival (tumor-free) function  of the three diet groups in order to obtain information about the distribution of the tumor-free time. The three survival functions were estimated using the Kaplan-Meier PL method.
We will proceed to test the significant difference among three diet group by using the K-Sample Gehan$'$s  Generalized Wilcoxon Test, Generalized Log rank and Peto-Peto Test. The SAS and R language is used to conduct the various test after which a comparison is made between them. If a significant difference is detected, we shall proceed with a paired comparison test to investigate which specific pairs are contributing to the significant difference. We update the new p-values of the paired comparison to account for the repeat testing in a $“$data dredging$”$ manner. Several methods exist to achieve this, among them are the Bonferroni, Fisher$'$s Least Significant Difference and Tukey-Kramer Methods

\begin{figure}[h!]
  \caption{Tumor-Free Times Rats on Three Different Diets.}
  \centering
    \includegraphics[width=17cm,height=16cm]{table34.png}
%\includegraphics[height=0.5\textheight]{project.png}
\label{fig:awesome_image}
\end{figure}
\newpage
From this table, we can easily tell that 15 rats are tumor-free in the low-fat group at the end of the study, and 7 rats are tumor-free in the saturated fat group.  In order to determine the effectiveness of a diet in keeping rats tumor-free, we need to analyze these data with three different tests and come to congruent answer. Without the methods used in survival analysis to compare censored and non-censored data, the number of tumor-free rats at the end of the study, or censored observations, in each group would be compared.  Using the survival analysis skills, the ranks of the censored and uncensored observations can be compared. 

First of all, we extended three tests for k-samples with Kruskal and Wallis$'$s test. The following are results from R and SAS


\section{STATISTICAL METHODS , ANALYSIS AND RESULTS}
\subsection{Analysis From R}
%\include{project.r}
\begin{verbatim}
 rm(list=ls())
library(survival)
#h0:g1=g2=g3  H0: the 3 groups survival curves are the same.
#ha:at at  least one survival curve is  different

low.fat<-c(140,177,50,65,86,153,181,191,77,84,87,56,66,73,199,140,200,200
,200,200,200,200,200,200,200,200,200,200,200,200)
length(low.fat)

s1<-c(1,1,1,1,1,1,1,1,1,1,1,1,1,1,1,0,0,0,0,0,0,0,0,0,0,0,0,0,0,0)

group.1=rep(1,length(low.fat))

saturated.fat<-c(124,58,56,68,79,89,107,86,142,110,96,142,86,75,117,98,105,126,43,
46,81,133,165,170,200,200,200,200,200,200)



s2<-c(rep(1,23),0,0,0,0,0,0,0)

group.2=rep(2,length(saturated.fat))

unsaturated.fat<-c(112,68,84,109,153,143,60,70,98,164,63,63,77,91,91,66,70,
77,63,66,66,94,101,105,108,112,115,126,161,178)



s3<-c(rep(1,length(unsaturated.fat)))

group.3=rep(3,length(unsaturated.fat))

time<-c(low.fat,saturated.fat,unsaturated.fat)

status<-c(s1,s2,s3)

group=c(group.1,group.2,group.3)

datatable3.3=data.frame(time,status,group)

 
 plot(survfit(Surv(time, status) ~ group, data=datatable3.3)
 , xlab="Time", ylab="Survival Probability",col=c("blue","red","green"), lwd = 3,
 lty=1:3, mark.time=FALSE)
title(main='KM-Curves for Tumor-Free Time of 90 Rats on Three Different Diets ') 
legend(150, 1,c('low.fat', 'saturated.fat','unsaturated.fat'), pch=c(1,2,3) 
,col=c("blue","red","green"))


 ##### K-sample Long-Rank test,rh0=0
 ##
 survdiff(Surv(time, status) ~ group, data=datatable3.3,rho=0)


##### K-sample Peto-Peto test,rho=1
 ##
 survdiff(Surv(time, status) ~ group, data=datatable3.3,rho=1)


\end{verbatim}

\subsection{Output From R}

\begin{verbatim}
 >  ##### K-sample Long-Rank test,rh0=0
>  ##
>  survdiff(Surv(time, status) ~ group, data=datatable3.3,rho=0)
Call:
survdiff(formula = Surv(time, status) ~ group, data = datatable3.3, 
    rho = 0)

         N Observed Expected (O-E)^2/E (O-E)^2/V
group=1 30       15     30.2    7.6291   14.7664
group=2 30       23     21.8    0.0627    0.0939
group=3 30       30     16.0   12.2555   17.1738

 Chisq= 21.8  on 2 degrees of freedom, p= 1.81e-05 
> 
> 
> ##### K-sample Peto-Peto test,rho=1
>  ##
>  survdiff(Surv(time, status) ~ group, data=datatable3.3,rho=1)
Call:
survdiff(formula = Surv(time, status) ~ group, data = datatable3.3, 
    rho = 1)

         N Observed Expected (O-E)^2/E (O-E)^2/V
group=1 30     9.16     17.4    3.9048    9.8426
group=2 30    14.60     14.1    0.0208    0.0442
group=3 30    19.19     11.5    5.1641    9.8152

 Chisq= 13.2  on 2 degrees of freedom, p= 0.00136 
> 
> 

\end{verbatim}

\begin{figure}[h!]
  \caption{Tumor-Free Times Rats on Three Different Diets.}
  \centering
    \includegraphics[width=17cm,height=15cm]{km1.png}
%\includegraphics[height=0.5\textheight]{project.png}
\label{fig:awesome_image}
\end{figure}


\subsection{Analysis From SAS}
\begin{verbatim}
ods html close; /* close previous */
ods html; /* open new */
*low-fat=1  saturated fat=2 unsaturated fat=3*;	


data exercisethree;
   
   input time  Status Treatment;

   datalines;
140 1 1
177 1  1
50 1 1
65   1 1
86 1 1
153 1 1
181 1 1
.    .     .
;

proc print data=exercisethree;
run;

PROC LIFETEST DATA=exercisethree;
TIME time*Status(0);
strata Treatment/TEST=(LOGRANK WILCOXON PETO);
run;
*low-fat=1  saturated fat=2 unsaturated fat=3*;

\end{verbatim}
\subsection{SAS Output}

\begin{verbatim}
 
Test of Equality over Strata
Test	Chi-Square	DF	Pr >
Chi-Square
Log-Rank	20.7404	2	<.0001
Wilcoxon	12.3992	2	0.0020
Peto	12.3755	2	0.0021




The LIFETEST Procedure

Testing Homogeneity of Survival Curves for time over Strata
Rank Statistics
Treatment	Log-Rank	Wilcoxon	Peto
1	-14.768	-714.00	-7.769
2	0.969	37.00	0.463
3	13.799	677.00	7.306

\end{verbatim}



\begin{figure}[h!]
  \caption{Tumor-Free Times Rats on Three Different Diets.}
  \centering
    \includegraphics[width=16cm,height=15cm]{km2.png}
%\includegraphics[height=0.5\textheight]{project.png}
\label{fig:awesome_image}
\end{figure}

\newpage
\subsection{Comparison of SAS And R Output}
\begin{center}
\scalebox{2}{
    \begin{tabular}{ | l | l | l | p{5cm} |}
    \hline
     & SAS & R  \\ \hline
   Gehan$'$s Wilcoxon & 0.0024 &0.0020  
     \\ \hline
    Log-rank &1.81e-05 & <.0001
    . \\ \hline
   Peto-Peto &0.00136& 0.0021  \\
    \hline
    \end{tabular}
}
\end{center}

\hspace{5mm}
Furthermore, after K-sample analysis, we did find difference within three groups. In order to find out which group is different from others, the paired comparisons between two the treatments are developed in SAS. Here are codes and results for paired comparison. 



\newpage
\subsection{PAIRED    COMPARISON BETWEEN TREATMENT  GROUPS}
For k samples,there are a total of $\left( 
\begin{array}{c} 
k\\ 
2 
\end{array} 
\right) =\frac{k(k-1)}{2}$ 
 pairwise comparisons. For k groups, the generalized Wilcoxon, Peto-peto and Log-rank test can be used to look for a difference across k group means as a whole. If there is a statistically significant difference across k means then a multiple comparison method can be used to look for specific differences between pairs of groups. The reason that two sample methods should not be used to make multiple pairwise comparisons is that they are not designed for repeat testing in a "data dredging" manner.
There are several methods for adjusting the significance level and p-values for the pairwise comparison test. Among them are 
For k samples , there is $\left( 
\begin{array}{c} 
k\\ 
2 
\end{array} 
\right)=    (k(k-1))/2$ number of pairwise comparisons. For k groups, the generalized Wilcoxon, Peto-peto and Log-rank test can be used to look for a difference across k group means as a whole. If there is a statistically significant difference across k means then a multiple comparison method can be used to look for specific differences between pairs of groups. The reason that two sample methods should not be used to make multiple pairwise comparisons is that they are not designed for repeat testing in a "data dredging" manner.
There are several methods for adjusting the significance level and p-values for the pairwise comparison test. Among them are 
\begin{itemize}
\item \textbf{	Bonferroni Adjustment}
The most flexible multiple comparisons procedure is the Bonferroni adjustment. In order to insure that the probability is no greater than α that something will appear to be statistically significant when there are no underlying differences, each of   $\frac{k(k-1)}{2}$  individual comparisons is performed at the   $\frac{\alpha}{\frac{k(k-1)}{2}} $   level of significance. Equivalently, each unadjusted p-value could also be multiplied   $\frac{k(k-1)}{2}$ and the result compared with the nominal significance level.
The biggest disadvantage of the Bonferroni method is that  it is not an exact test.

\item 	\textbf{Fisher$'$s Protected Least Significant Difference(LSD)}
We start by performing an overall F-ratio Test. If the overall F-ratio test with the null hypothesis that all group means are equal is statistically significant, we can confidently conclude that not all of the treatment means are identical. After this we can carry out all possible  $\frac{k(k-1)}{2}$   t tests.
\item\textbf{\color{black} Tukeys  Honest Significant Difference(HSD) Method}\\
The normalized mean difference between the groups I and j is given as;

 \begin{equation*}
\scalebox{1.8}{
$T _{ij}=\frac{|\bar{x}_{i}-\bar{x}_{j}|}{\sqrt{\frac{MSE}{2}\left(\frac{1}{n_{i}}+\frac{1}{n_{j}}\right)}}>q(\alpha,k,df)$}
\end{equation*}

\end{itemize}

Pairwise comparisons with equal group sizes can be done with the very conservative Tukey$'$s HSD whereas unequal group sizes can be done with Tukey-Kramer or Scheff$\acute{e}$ methods
PROC LIFETEST in SAS performs the paired comparisons based on Tukey$'$s studentized range test. PROC LIFETEST uses the approximation described in Kramer (1956) and identifies the adjustment as "Tukey-Kramer" in the results. Note that ADJUST$=$TUKEY is incompatible with DIFF=CONTROL [2]. Tukey (-Kramer), consider the statistical distributions associated with systematic repeated testing. The most acceptable general method for all pairwise comparisons is ( Tukey -Kramer), the P values for which are exact with balanced designs (Hsu, 1996).















\begin{verbatim}
data exercisethree;
   
   input time  Status Treatment;

   datalines;
140 1 1
177 1  1
50 1 1
65   1 1
86 1 1
. .  .
86 1 2;

proc print data=exercisethree;
run;

PROC LIFETEST DATA=exercisethree;
TIME time*Status(0);
strata Treatment/TEST=(LOGRANK WILCOXON PETO) diff=all  ADJUST=TUKEY;
* By specifying diff=all in the strata statement we get the multiple  comparison between 
*all the groups;
* Adjust =Tukey specifies the Tukey-Kramer adjustment of the p-values;

run;


\end{verbatim}


\begin{verbatim}
Adjustment for Multiple Comparisons for the
Logrank Test
Strata Comparison	Chi-Square	p-Values
Treatment	Treatment		Raw	Tukey-Kramer
1	2	5.0446	0.0247	0.0637
1	3	20.4685	<.0001	<.0001
2	3	4.4733	0.0344	0.0868

Adjustment for Multiple Comparisons for the
Wilcoxon Test
Strata Comparison	Chi-Square	p-Values
Treatment	Treatment		Raw	Tukey-Kramer
1	2	3.3208	0.0684	0.1623
1	3	12.3889	0.0004	0.0013
2	3	2.7401	0.0979	0.2226

Adjustment for Multiple Comparisons for the
Peto Test
Strata Comparison	Chi-Square	p-Values
Treatment	Treatment		Raw	Tukey-Kramer
1	2	3.3853	0.0658	0.1567
1	3	12.3689	0.0004	0.0013
2	3	2.6634	0.1027	0.2321


\end{verbatim}

\newpage
\section{Conclusion}
Our goal in this project was to determine the ability of three diets to keep rats tumor free estimating survival function which is obtained from information about the distribution of the tumor-free time of rats in the three diet groups. 
The p-values from the log-rank and Peto-Peto which are 1.81e-05 and 0.00136 respectively, This leads us to make the decision to reject the null hypothesis that all three tumor-free times produced by the three rats   are equal at a significance level of 0.05. We conclude that the data show a significant difference among the tumor-free times produced by the three diets.
We proceeded to test which particular pairs were contributing to the significant difference. Based on our SAS output ,the raw p-values which is not adjusted for paired comparison based on the Log-rank test  between treatment groups  1 and 2,1 and 3 , 2 and 3 are 0.0247,less than 0.0001 and 0.0344 respectively. This indicates significant difference between all groups at a significance level of 0.05.On the other hand the p-values adjusted by the Tukey-Kramer method  indicates a significant difference between only groups 1 and 3 at a significance level of 0.05.This can be explained by the rather high number of censored data in group1 and no censored data in group 3.There is a relatively fair amount of censored data in both group 1 and 2 ,this accounts for non-detection  difference  between the two groups.

The comparison of the output by R and SAS revealed that, they do not churn out exactly the same values but are significantly close. The result from both output leads to consistent conclusions. The difference could be attributed to the difference in algorithms used to implement the statistical procedures in both programming languages.



\newpage
\begin{thebibliography}{100}

\bibitem{Lee}
Statistical Methods For Survival Data Analysis, Third Edition, Elisa T. Lee and John Wenyu Wang

%\bibitem{sas}
%begin{verbatim}
%http://support.sas.com/documentation/cdl/en/statug/63347/HTML/default/viewer.htm#statug_lifetest_sect008.htm
%\end{verbatim}
%\bibitem{three}

%begin{verbatim}
%http://www.statsdirect.com/help/default.htm#analysis_of_variance/multiple_comparisons.htm
%\end{verbatim}
\end{thebibliography}


\newpage

\section{Appendices}
\begin{verbatim}
 
The SAS System

The LIFETEST Procedure
Stratum 1: Treatment = 1
Product-Limit Survival Estimates
time		Survival	Failure	Survival Standard
Error	Number
Failed	Number
Left
0.000		1.0000	0	0	0	30
50.000		0.9667	0.0333	0.0328	1	29
56.000		0.9333	0.0667	0.0455	2	28
65.000		0.9000	0.1000	0.0548	3	27
66.000		0.8667	0.1333	0.0621	4	26
73.000		0.8333	0.1667	0.0680	5	25
77.000		0.8000	0.2000	0.0730	6	24
84.000		0.7667	0.2333	0.0772	7	23
86.000		0.7333	0.2667	0.0807	8	22
87.000		0.7000	0.3000	0.0837	9	21
119.000		0.6667	0.3333	0.0861	10	20
140.000		0.6333	0.3667	0.0880	11	19
140.000	*	.	.	.	11	18
153.000		0.5981	0.4019	0.0899	12	17
177.000		0.5630	0.4370	0.0912	13	16
181.000		0.5278	0.4722	0.0920	14	15
191.000		0.4926	0.5074	0.0924	15	14
200.000	*	.	.	.	15	13
200.000	*	.	.	.	15	12
200.000	*	.	.	.	15	11
200.000	*	.	.	.	15	10
200.000	*	.	.	.	15	9
200.000	*	.	.	.	15	8
200.000	*	.	.	.	15	7
200.000	*	.	.	.	15	6
200.000	*	.	.	.	15	5
200.000	*	.	.	.	15	4
200.000	*	.	.	.	15	3
200.000	*	.	.	.	15	2
200.000	*	.	.	.	15	1
200.000	*	0.4926	0.5074	.	15	0

Note:	The marked survival times are censored observations.

Summary Statistics for Time Variable time
Quartile Estimates
Percent	Point
Estimate	95% Confidence Interval
		Transform	[Lower	Upper)
75	.	LOGLOG	.	.
50	191.000	LOGLOG	119.000	.
25	86.000	LOGLOG	65.000	153.000

Mean	Standard Error
148.885	10.137

Note:	The mean survival time and its standard error were underestimated because the largest observation was censored and the estimation was restricted to the largest event time.

________________________________________
The SAS System

The LIFETEST Procedure
Stratum 2: Treatment = 2
Product-Limit Survival Estimates
time		Survival	Failure	Survival Standard
Error	Number
Failed	Number
Left
0.000		1.0000	0	0	0	30
43.000		0.9667	0.0333	0.0328	1	29
46.000		0.9333	0.0667	0.0455	2	28
56.000		0.9000	0.1000	0.0548	3	27
58.000		0.8667	0.1333	0.0621	4	26
68.000		0.8333	0.1667	0.0680	5	25
75.000		0.8000	0.2000	0.0730	6	24
79.000		0.7667	0.2333	0.0772	7	23
81.000		0.7333	0.2667	0.0807	8	22
86.000		.	.	.	9	21
86.000		0.6667	0.3333	0.0861	10	20
89.000		0.6333	0.3667	0.0880	11	19
96.000		0.6000	0.4000	0.0894	12	18
98.000		0.5667	0.4333	0.0905	13	17
105.000		0.5333	0.4667	0.0911	14	16
107.000		0.5000	0.5000	0.0913	15	15
110.000		0.4667	0.5333	0.0911	16	14
117.000		0.4333	0.5667	0.0905	17	13
124.000		0.4000	0.6000	0.0894	18	12
126.000		0.3667	0.6333	0.0880	19	11
133.000		0.3333	0.6667	0.0861	20	10
142.000		.	.	.	21	9
142.000		0.2667	0.7333	0.0807	22	8
165.000		0.2333	0.7667	0.0772	23	7
170.000	*	.	.	.	23	6
200.000	*	.	.	.	23	5
200.000	*	.	.	.	23	4
200.000	*	.	.	.	23	3
200.000	*	.	.	.	23	2
200.000	*	.	.	.	23	1
200.000	*	0.2333	0.7667	.	23	0

Note:	The marked survival times are censored observations.

Summary Statistics for Time Variable time
Quartile Estimates
Percent	Point
Estimate	95% Confidence Interval
		Transform	[Lower	Upper)
75	165.000	LOGLOG	124.000	.
50	108.500	LOGLOG	86.000	142.000
25	81.000	LOGLOG	56.000	96.000

Mean	Standard Error
112.900	7.466

Note:	The mean survival time and its standard error were underestimated because the largest observation was censored and the estimation was restricted to the largest event time.

________________________________________
The SAS System

The LIFETEST Procedure
Stratum 3: Treatment = 3
Product-Limit Survival Estimates
time		Survival	Failure	Survival Standard
Error	Number
Failed	Number
Left
0.000		1.0000	0	0	0	30
60.000		0.9667	0.0333	0.0328	1	29
63.000		.	.	.	2	28
63.000		.	.	.	3	27
63.000		0.8667	0.1333	0.0621	4	26
66.000		.	.	.	5	25
66.000		.	.	.	6	24
66.000		0.7667	0.2333	0.0772	7	23
68.000		0.7333	0.2667	0.0807	8	22
70.000		.	.	.	9	21
70.000		0.6667	0.3333	0.0861	10	20
77.000		.	.	.	11	19
77.000		0.6000	0.4000	0.0894	12	18
84.000		0.5667	0.4333	0.0905	13	17
91.000		.	.	.	14	16
91.000		0.5000	0.5000	0.0913	15	15
94.000		0.4667	0.5333	0.0911	16	14
98.000		0.4333	0.5667	0.0905	17	13
101.000		0.4000	0.6000	0.0894	18	12
105.000		0.3667	0.6333	0.0880	19	11
108.000		0.3333	0.6667	0.0861	20	10
109.000		0.3000	0.7000	0.0837	21	9
112.000		.	.	.	22	8
112.000		0.2333	0.7667	0.0772	23	7
115.000		0.2000	0.8000	0.0730	24	6
126.000		0.1667	0.8333	0.0680	25	5
143.000		0.1333	0.8667	0.0621	26	4
153.000		0.1000	0.9000	0.0548	27	3
161.000		0.0667	0.9333	0.0455	28	2
164.000		0.0333	0.9667	0.0328	29	1
178.000		0	1.0000	.	30	0

Summary Statistics for Time Variable time
Quartile Estimates
Percent	Point
Estimate	95% Confidence Interval
		Transform	[Lower	Upper)
75	112.000	LOGLOG	101.000	153.000
50	92.500	LOGLOG	70.000	109.000
25	68.000	LOGLOG	63.000	77.000

Mean	Standard Error
98.467	6.193

Summary of the Number of Censored and Uncensored Values
Stratum	Treatment	Total	Failed	Censored	Percent
Censored
1	1	30	15	15	50.00
2	2	30	23	7	23.33
3	3	30	30	0	0.00
Total		90	68	22	24.44

________________________________________
The SAS System

The LIFETEST Procedure

Testing Homogeneity of Survival Curves for time over Strata
Rank Statistics
Treatment	Log-Rank	Wilcoxon	Peto
1	-14.768	-714.00	-7.769
2	0.969	37.00	0.463
3	13.799	677.00	7.306

Covariance Matrix for the Log-Rank
Statistics
Treatment	1	2	3
1	15.6801	-9.3772	-6.3029
2	-9.3772	14.6563	-5.2791
3	-6.3029	-5.2791	11.5821

Covariance Matrix for the Wilcoxon
Statistics
Treatment	1	2	3
1	55839.7	-30196.9	-25642.8
2	-30196.9	53607.4	-23410.5
3	-25642.8	-23410.5	49053.3

Covariance Matrix for the Peto
Statistics
Treatment	1	2	3
1	6.57894	-3.56412	-3.01482
2	-3.56412	6.31396	-2.74984
3	-3.01482	-2.74984	5.76467

Test of Equality over Strata
Test	Chi-Square	DF	Pr >
Chi-Square
Log-Rank	20.7404	2	<.0001
Wilcoxon	12.3992	2	0.0020
Peto	12.3755	2	0.0021


 


\end{verbatim}

\end{document}
