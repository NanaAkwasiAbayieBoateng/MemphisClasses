\documentclass[hyperref={pdfpagelabels=false}]{beamer}
% By  using hyperref={pdfpagelabels=false} you get rid off:
% Package hyperref Warning: Option `pdfpagelabels' is turned off
% (hyperref)                because \thepage is undefined. 
% Hyperref stopped early 
%

\usepackage{lmodern}
% Using lmondern and you get rid off this:
% LaTeX Font Warning: Font shape `OT1/cmss/m/n' in size <4> not available
% (Font)              size <5> substituted on input line 22.
% LaTeX Font Warning: Size substitutions with differences
% (Font)              up to 1.0pt have occurred.
%
\usepackage{hyperref}

% If \titel{$B!D(B} \author{$B!D(B} come after \begin{document} 
% you get the following warnig:
% Package hyperref Warning: Option `pdfauthor' has already been used,
% (hyperref) ... 
% So it is here before \begin{document}

\usepackage[T1]{fontenc}

\usepackage{fix-cm}

\title{PAIRED    COMPARISON BETWEEN  TREATMENT  GROUPS}   
\author{Nana Boateng \\
 Bo Jiang\\
 William Terry \\
Jingnan Zhao} 
\date{\today} 
% additional usepackage{beamerthemeshadow} is used

\usepackage{beamerthemeshadow}
%\usetheme{CambridgeUS}
    \usepackage[latin1]{inputenc}
    \usefonttheme{professionalfonts}
    \usepackage{times}
    \usepackage{tikz}
    \usepackage{amsmath}
    \usepackage{verbatim}
    \usepackage[T1]{fontenc}
    \usepackage{mathtools}
    \usepackage{fix-cm}
    \usetikzlibrary{arrows,shapes}
    \DeclareMathOperator*{\Max}{Max}

\usepackage{graphicx}% http;//ctan.org/pkg/graphicx
\usepackage{amsfonts}
\usepackage{amssymb}
\usepackage{amsmath}
\usepackage{fancyvrb}
\usepackage{listings}
\usepackage{lstautogobble}

\lstset{basicstyle=\ttfamily,
  mathescape=true,
  escapeinside=||,
  autogobble}
% Let us first define a custom Verbatim environment, that saves us a lot of writing
  \DefineVerbatimEnvironment{VerbatimTest}{Verbatim}%
    {showspaces,showtabs,commandchars=\\\{\}}% showspaces and showtabs only for visualizing
\lstdefinestyle{customc}{
  belowcaptionskip=1\baselineskip,
  breaklines=true,
  frame=L,
  xleftmargin=\parindent,
  language=SAS,
  showstringspaces=false,
  basicstyle=\footnotesize\ttfamily,
  keywordstyle=\bfseries\color{green!40!black},
  commentstyle=\itshape\color{purple!40!black},
  identifierstyle=\color{blue},
  stringstyle=\color{orange},
}

\lstdefinestyle{customasm}{
  belowcaptionskip=1\baselineskip,
  frame=L,
  xleftmargin=\parindent,
  language=[x86masm]Assembler,
  basicstyle=\footnotesize\ttfamily,
  commentstyle=\itshape\color{purple!40!black},
}

\lstset{escapechar=@,style=customc}
\lstset{language=SAS,caption={Descriptive Caption Text},label=DescriptiveLabel}
\usepackage{listings}
\usepackage{color}

\definecolor{mygreen}{rgb}{0,0.6,0}
\definecolor{mygray}{rgb}{0.5,0.5,0.5}
\definecolor{mymauve}{rgb}{0.58,0,0.82}

\lstset{ %
  backgroundcolor=\color{white},   % choose the background color; you must add \usepackage{color} or \usepackage{xcolor}
  basicstyle=\footnotesize,        % the size of the fonts that are used for the code
  breakatwhitespace=false,         % sets if automatic breaks should only happen at whitespace
  breaklines=true,                 % sets automatic line breaking
  captionpos=b,                    % sets the caption-position to bottom
  commentstyle=\color{mygreen},    % comment style
  deletekeywords={...},            % if you want to delete keywords from the given language
  escapeinside={\%*}{*)},          % if you want to add LaTeX within your code
  extendedchars=true,              % lets you use non-ASCII characters; for 8-bits encodings only, does not work with UTF-8
  frame=single,                    % adds a frame around the code
  keepspaces=true,                 % keeps spaces in text, useful for keeping indentation of code (possibly needs columns=flexible)
  keywordstyle=\color{blue},       % keyword style
  language=Octave,                 % the language of the code
  morekeywords={*,...},            % if you want to add more keywords to the set
  numbers=left,                    % where to put the line-numbers; possible values are (none, left, right)
  numbersep=5pt,                   % how far the line-numbers are from the code
  numberstyle=\tiny\color{mygray}, % the style that is used for the line-numbers
  rulecolor=\color{black},         % if not set, the frame-color may be changed on line-breaks within not-black text (e.g. comments (green here))
  showspaces=false,                % show spaces everywhere adding particular underscores; it overrides 'showstringspaces'
  showstringspaces=false,          % underline spaces within strings only
  showtabs=false,                  % show tabs within strings adding particular underscores
  stepnumber=2,                    % the step between two line-numbers. If it's 1, each line will be numbered
  stringstyle=\color{mymauve},     % string literal style
  tabsize=2,                       % sets default tabsize to 2 spaces
  title=\lstname                   % show the filename of files included with \lstinputlisting; also try caption instead of title
}


\usepackage{tabularx}
\usepackage{multirow}
\usepackage{color}





%\titleformat{\section}[block]{\color{black}\Large\bfseries\filcenter}{}{1em}{}
%\titleformat{\subsection}[hang]{\bfseries}{}{1em}{}

\setcounter{secnumdepth}{1}

% allows use of "@" in control sequence names
\makeatletter
\DeclareGraphicsExtensions{.pdf,.png,.jpg}
% this creates a custom and simpler ruled box style
\newcommand\floatc@simplerule[2]{{\@fs@cfont #1 #2}\par}
\newcommand\fs@simplerule{\def\@fs@cfont{\bfseries}\let\@fs@capt\floatc@simplerule
  \def\@fs@pre{\hrule height.8pt depth0pt \kern4pt}%
  \def\@fs@post{\kern4pt\hrule height.8pt depth0pt \kern4pt \relax}%
  \def\@fs@mid{\kern8pt}%
  \let\@fs@iftopcapt\iftrue}





\def\aa{\vec{a}}
\def\bb{\vec{b}}
\def\cc{\vec{c}}
\def\ee{\vec{e}}
\def\pp{\vec{p}}
\def\xx{\vec{x}}
\def\yy{\vec{y}}
\def\zz{\vec{z}}
\def\PP{\vec{P}}
\def\SS{\vec{S}}
\def\aalpha{\vec{\alpha}}
\def\bbeta{\vec{\beta}}
\def\ggamma{\vec{\gamma}}
\def\llambda{\vec{\lambda}}
\def\nnu{\vec{\nu}}

\def\cD{\mathcal{D}}
\def\cF{\mathcal{F}}
\def\cG{\mathcal{G}}
\def\cI{\mathcal{I}}
\def\cL{\mathcal{L}}
\def\cM{\mathcal{M}}
\def\cN{\mathcal{N}}
\def\cO{\mathcal{O}}
\def\cP{\mathcal{P}}
\def\cQ{\mathcal{Q}}
\def\cS{\mathcal{S}}
\def\cX{\mathcal{X}}
%\setbeamertemplate{footline}{\insertframenumber/\inserttotalframenumber}
%\useoutertheme{infolines}
%\usetheme{Hannover}
%\setcounter{captureframe}{\value{framenumber}}
\mode<presentation>
 {
   \usetheme{Dresden}
 }

 \setbeamertemplate
 {footline}{\quad\hfill\insertframenumber/\inserttotalframenumber\strut\quad} 



\begin{document}

\begin{frame}
\titlepage
\end{frame} 

\begin{frame}
\frametitle{Table of contents}
\tableofcontents
\end{frame}
 


\section{Methods of Adjusting P-values} 


\begin{frame}
\frametitle{\textbf{Methods of Adjusting P-values}}
For k samples,there are a total of $\left( 
\begin{array}{c} 
k\\ 
2 
\end{array} 
\right) =\frac{k(k-1)}{2}$ 
pairwise comparisons
\pause
\begin{itemize}
\item\textbf{\color{blue} Bonferroni Method}\\


% \begin{equation*}
\scalebox{2}{
$\alpha_{new}=\frac{\alpha}{\frac{k(k-1)}{2}}$}
%\end{equation*}
%\end{itemize}
%\end{frame}
\pause
%\begin{itemize}
\item\textbf{\color{blue} Fishers Protected Least Significant Difference Method}\\
\begin{itemize}
\item perform overall F-Test for significant difference.
\item If significant perform$ \frac{k(k-1)}{2}$ pairewise t-test
\item Fisher's LSD does not correct for multiple comparisons.
\end{itemize}
\end{itemize}
\end{frame}	

\begin{frame}
\begin{itemize}

\item\textbf{\color{blue} Tukeys  Honest Significant Difference(HSD) Method}\\
The normalized mean difference between the groups I and j is given as;
\pause
 \begin{equation*}
\scalebox{1.8}{
$T _{ij}=\frac{|\bar{x}_{i}-\bar{x}_{j}|}{\sqrt{\frac{MSE}{2}\left(\frac{1}{n_{i}}+\frac{1}{n_{j}}\right)}}>q(\alpha,k,df)$}
\end{equation*}
\end{itemize}
\end{frame}





\section{SAS CODES} 

\begin{frame}
\frametitle{SAS CODES}
\begin{figure}[h!]
%  \caption{SAS CODES}
  \centering
    \includegraphics[width=10cm,height=3cm]{sas.png}
%\includegraphics[height=0.5\textheight]{project.png}
%\label{fig:awesome_image}
\end{figure} 
\end{frame}
\begin{frame}
\frametitle{SAS OUTPUT}
\begin{figure}[h!]
%  \caption{SAS OUTPUT}
  \centering
    \includegraphics[width=10cm,height=10cm]{results1.png}
%\includegraphics[height=0.5\textheight]{project.png}
%\label{fig:awesome_image}
\end{figure}

\end{frame}



%\subsection{Lists I}






\begin{frame}
\frametitle{Summary of Data}
\begin{table}[h]
\centering
\begin{tabular}{ |l|l|l|l| }
\multicolumn{3}{c}{} \\
\cline{1-4}
  & Low Fat  & Saturated Fat&Unsaturated \\
&  & &Fat\\
\hline
Number of &    15  & 7   & 0     \\
  Censored observations &  &   &\\
               \hline
    Number of &  15    & 23  & 30     \\
  Events & &  &\\
               \hline

\end{tabular}
\caption{Summary of Data }
\end{table}
\end{frame}

\begin{frame}
\frametitle{Conclusion}
\begin{itemize}
\item In this study we were interested in investigating relationship between diet and development of tumors
\item We  estimated the survival (tumor-free) function  of the three diet groups in order to obtain information about the distribution of the tumor-free time.
\item We tested for overall significant difference among the three groups using K-Sample Gehans Generalized Wilcoxon Test, Generalized Log rank and Peto-Peto Test.
\item After finding a significant difference between the groups,we proceeded to investigate which paired group was contributing  significantly to the diference and found groups 1 and 3 to be so.

\end{itemize}



\end{frame}







%\section{Section no. 4}
%\subsection{blocs}
\begin{frame}
%\frametitle{blocs}

\begin{block}{}
Thank you for your attention.Any questions?
\end{block}

%\begin{exampleblock}{title of the bloc}
%bloc text
%\end{exampleblock}


%\begin{alertblock}{title of the bloc}
%%bloc text
%\end{alertblock}
\end{frame}
\begin{frame}
%\href{run:survival.pdf}{http://yihui.name/en/2008/10/how-to-open-a-file-from-a-hyperlink/}
\hyperlink{frame6}{\beamergotobutton{Go to frame 6}}	
\end{frame}

\begin{frame}
	{\footnotesize
		\bibliographystyle{authortitle1}
		\bibliography{TEST}
	}
\end{frame}

\begin{frame}[allowframebreaks]
	\frametitle{References}
	\bibliographystyle{amsalpha}
	\bibliography{../bib_files/jabrefmaster.bib}
	\begin{center}
		{\bf BIBLIOGRAPHY}
	\end{center}
	\addcontentsline{toc}{section}{\rm BIBLIOGRAPHY \dotfill}
	\begin{thebiblio}{}
		%---------------------------------------------------------------------------------------------------------------------------
		
		
		
		\bibitem{BS73}
		F. Black, M.S. Scholes, \emph{The pricing of options and corporate
			liabilities}, Journal of Political Economy 81 (1973) 637-659
		\bibitem{Fas02}
		G.E. Fasshauer, A.Q.M. Khaliq, D.A. Voss, \emph{Using meshfree
			approximation for multi-asset American option problems}, Journal of
		Chinese Institute of Engineers, 27 (2004) 563-571
		
		\bibitem{NB11}
		N. Flyer and B. Fornberg,\emph{ Radial basis function: Development
			and applications to planetary scale flows, Computers and Fluids}
		46(2011) 23-32.
		\bibitem{BEN11}
		B. Fornberg, E. Larsson, and N. Flyer,\emph{ Stable computations
			with Gaussian Radial Basis Functions}, SIAM Journal on Scientific
		Computing, 33 (2011), 869-892.
		\bibitem{YZE07}
		Y. Goto, Z. Fei, S.Kan, and E. Kita, \emph{Options valuation by
			using radial basis function approximation}, Engineering Analysis
		with Boundary Elements, 31(2007) 836-843.
		\bibitem{Hon09}
		Y.C. Hon and Z. Yang, \emph{Meshless collocation method by
			Delta-shaped basis functions for default barrier model}, Engineering
		Analysis with Boundary Elements 33(2009) 951-958.
		\bibitem{ADS06}
		A.Q.M. Khaliq, D.A. Voss, S.H.K. Kazmi,   \emph{A linearly implicit
			predictor-corrector scheme for pricing American options using a
			penalty method approach}, Journal of Banking and Finance,30 (2006)
		489-502.
		\bibitem{BSA02}
		B. Nielsen, O. Skavhaug, and A. Tveito, \emph{Penalty and
			front-fixing methods for the numerical solution of American option
			problems}, Journal of Computational Finance, 5 (2002) 69-97.
		\bibitem{BOA08}
		B. Nielsen, O. Skavhaug, and A. Tveito,\emph{ Penalty methods for
			the numerical solution of American multi-asset option problems},
		Journal of Computational and Applied Mathematics, 222 (2008) 3-16.
		\bibitem{UEG}
		U.  Pettersson, E. Larsson, G.  Marcusson, and J. Persson,
		\emph{Improved radial basis function methods for multi-dimensional
			option pricing}, Journal of Computational and Applied Mathematics,
		222 (2008) 82-93
		\bibitem{SE09}
		S. A. Sarra and E. J. Kansa,  \emph{Multiquadric Radial Basis
			Function Approximation Methods for the Numerical Solution of Partial
			Differential Equations}. Advances in Computational Mechanics, Tech
		Series Press, 2 (2009).
		%\bibitem{PSJ}
		%Paul Wilmott,Sam Howson,Jeff Dewynne  \emph{The Mathematics of
		%Financial Derivatives}. A Student Introduction, Cambridge university
		%press.
		\bibitem{higham}
		Desmond J. Higham  \emph{An Introduction to Financial Option
			Valuation}. Mathematics,Stochchastics and Computation,cambridge
		univeristy press(2004).
		\bibitem{CF}
		Omur Ogur \emph{An Introduction to Computational Finance}.Imperial
		College press
		\bibitem{Wilmot}
		Paul Wilmot,Sam Howson,Jeff Dewynne\emph{The Mathematics of
			Financial Derivatives}:A Student Introduction.Cambridge University
		press
		\bibitem{Hon}
		Yiu-Chung Hon,Xian-Zhang Mao\emph{A Radial Basis Function Method for
			Solving Options Pricing Models}.The Journal of Finacial
		Engineering.(1999)
		
		\bibitem{KK}
		A.Q.M. Khaliq, D.A. Voss,Greg E.Fasshauer\emph{A Parallel Time
			Stepping Approach using Mesh-free Approximations for Pricing Options
			with Non-Smooth Payouts}:A Student Introduction.The Journal of
		Risk(2009)
	\end{thebiblio}
\end{frame}
\end{document}
