\documentclass[11pt]{article}
\usepackage{amsfonts}
\usepackage{amsmath}
\usepackage{color}
\usepackage{tabularx}
\usepackage{multirow}
\usepackage{color}
\usepackage[pdftex]{graphicx}
\usepackage{caption}
\usepackage{chngcntr}
\counterwithin{table}{section}
\usepackage{float}
\floatstyle{boxed}
\restylefloat{figure}
\usepackage[english]{babel}
\usepackage[pdftex]{hyperref}
\usepackage{xcolor}
\usepackage{titlesec}
\usepackage{lipsum}
\usepackage{lmodern}
\usepackage[T1]{fontenc}
\usepackage{textcomp}
\usepackage{fixltx2e}


%\titleformat{\section}[block]{\color{black}\Large\bfseries\filcenter}{}{1em}{}
%\titleformat{\subsection}[hang]{\bfseries}{}{1em}{}

\setcounter{secnumdepth}{1}

% allows use of "@" in control sequence names
\makeatletter
\DeclareGraphicsExtensions{.pdf,.png,.jpg}
% this creates a custom and simpler ruled box style
\newcommand\floatc@simplerule[2]{{\@fs@cfont #1 #2}\par}
\newcommand\fs@simplerule{\def\@fs@cfont{\bfseries}\let\@fs@capt\floatc@simplerule
  \def\@fs@pre{\hrule height.8pt depth0pt \kern4pt}%
  \def\@fs@post{\kern4pt\hrule height.8pt depth0pt \kern4pt \relax}%
  \def\@fs@mid{\kern8pt}%
  \let\@fs@iftopcapt\iftrue}

% this code block defines the new and custom floatbox float environment
\floatstyle{simplerule}
\newfloat{floatbox}{thp}{lob}[section]
\floatname{floatbox}{Text Box}

\usepackage{dcolumn}
%here we're setting up a version of the math fonts with normal x-width
\DeclareMathVersion{nxbold}
\SetSymbolFont{operators}{nxbold}{OT1}{cmr} {b}{n}
\SetSymbolFont{letters}  {nxbold}{OML}{cmm} {b}{it}
\SetSymbolFont{symbols}  {nxbold}{OMS}{cmsy}{b}{n}

\oddsidemargin .27in \evensidemargin .3in \textwidth 6.5in
\topmargin -.25in \textheight 8.75in

\newtheorem{thm}{Theorem}[section]
\newtheorem{cor}[thm]{Corollary}
\newtheorem{lem}[thm]{Lemma}
\newtheorem{prop}[thm]{Proposition}
\newtheorem{defn}[thm]{Definition}
\newtheorem{rem}[thm]{Remark}
\numberwithin{figure}{section}
\renewcommand{\thefootnote}{\fnsymbol{footnote}}

\def\RR{\mathbb{R}}
\def\NN{\mathbb{N}}
\def\ZZ{\mathbb{Z}}
\def\CC{\mathbb{C}}
\def\st{\mathop{\: | \:}\nolimits}

\def\vec#1{\mathchoice{\mbox{\boldmath$\displaystyle#1$}}
{\mbox{\boldmath$\textstyle#1$}}
{\mbox{\boldmath$\scriptstyle#1$}}
{\mbox{\boldmath$\scriptscriptstyle#1$}}}

\def\aa{\vec{a}}
\def\bb{\vec{b}}
\def\cc{\vec{c}}
\def\ee{\vec{e}}
\def\pp{\vec{p}}
\def\xx{\vec{x}}
\def\yy{\vec{y}}
\def\zz{\vec{z}}
\def\PP{\vec{P}}
\def\SS{\vec{S}}
\def\aalpha{\vec{\alpha}}
\def\bbeta{\vec{\beta}}
\def\ggamma{\vec{\gamma}}
\def\llambda{\vec{\lambda}}
\def\nnu{\vec{\nu}}

\def\cD{\mathcal{D}}
\def\cF{\mathcal{F}}
\def\cG{\mathcal{G}}
\def\cI{\mathcal{I}}
\def\cL{\mathcal{L}}
\def\cM{\mathcal{M}}
\def\cN{\mathcal{N}}
\def\cO{\mathcal{O}}
\def\cP{\mathcal{P}}
\def\cQ{\mathcal{Q}}
\def\cS{\mathcal{S}}
\def\cX{\mathcal{X}}

\title{\bf MATH 8742 Survival Analysis Final Report 
}
\author{Nana.~Boateng\footnotemark[1], Will .Terry\footnotemark[2], Jiang\footnotemark[3] and Jingnan
Zhao\footnotemark[7]}


\date{\today}

\begin{document}
\newcolumntype{d}{D{.}{.}{-1} } %decimal column as before
%wide bold decimal column
\newcolumntype{B}[3]{>{\boldmath\DC@{#1}{#2}{#3} }c<{\DC@end} }
%normal width bold decimal column
\newcolumntype{Z}[3]{>{\mathversion{nxbold}\DC@{#1}{#2}{#3} }c<{\DC@end} }
\makeatother
\maketitle \thispagestyle{empty}
\begin{abstract}
%\begin{center}
%\section{Abstract}
\end{abstract}

. 


%\end{abstract}
KEYWORDS:.
\fnsymbol{footnote}

\footnotetext[1]{Department of  Mathematics, University of Memphis}
\footnotetext[2]{Department of  Mathematics, University of Memphis}

 %\begin{center}
\section{Introduction}
%\end{center}







%\begin{center}
\section{Methodology}
%\end{center}

%\begin{center}
\section{Statistical Analysis And Results}
%\end{center}
\begin{table}[h]
\centering
\begin{tabular}{ |l|l|l| }

\multicolumn{2}{c}{} \\
\cline{1-3}
 Method   & P-value & Confidence Interval \\
& &For The Mean \\
\hline
T-test      & 0.0001252    & 0.4298118 ,1.1801882     \\
                 \hline
Wilcox.test       & 0.0007788    & 0.3899824 ,1.2150027      \\
\hline
Paired Comparison      & 3e-04    &       \\
 permutation Test &&\\
\hline
Binomial Sign Test & 0.0002012   &  0.7173815 ,1.0000000     \\
\hline
Bootstrap(BCa) & &0.4903,  1.1529\\
\hline

\end{tabular}
\caption{Table of Results of Various Statistical Test.}
\end{table}
\newpage


%\begin{figure}[h!]
 % \caption{Histogram and Quantile Plots of Sample and  Bootstrap Estimates.}
 % \centering
   % \includegraphics[width=17cm,height=15cm]{project.png}
%\includegraphics[height=0.5\textheight]{project.png}
%\label{fig:awesome_image}
%\end{figure}


%\begin{center}
\section{Discussion of Results}
%\end{center}


%\begin{center}
\section{Conclusion}
%\end{center}






\begin{thebibliography}{100}
\bibitem{B1}
	Davison, A.C. and Hinkley, D.V. (1997) Bootstrap Methods and Their Application. Cambridge University Press.
\bibitem{B2}
	http://www.webmd.com/hiv-aids/cd4-count-what-does-it-mean

\end{thebibliography}
\newpage
%\begin{center}
\section{Appendices}
%\end{center}




\end{document}
